\subsection*{Definitionen}
\begin{description}
  \item [Transponierte Matrix:]
    Sei $A \in \mathbb{R}^{m \times n}$ eine Matrix. Die transponierte Matrix 
    $A^t \in \mathbb{R}^{m \times n}$ entsteht, indem man Zeilen und Spalten 
    vertauscht: $a^t_{ij} := a_{ji}$.

  \item [Quadratisch:]
    Matrizen, die genau so viele Zeilen wie Spalten haben, heißen quadratisch.

  \item [Invertierbar:]
    Eine Matrix $A$ heißt invertierbar, wenn die zu $A$ gehörige lineare
    Abbildung $f$ (mit $f(\vec{x}) := A\vec{x}$) bijektiv ist 
    (d.h. Isomorphismus).\\
    Die Matrix der Umkehrabbildung $d^{-1}$ heißt inverse Matrix $A^{-1}$.

  \item [Determinante:]
    Eine Determinante ist eine Funktion 
    $det: \mathbb{K}^{n \times n} \rightarrow \mathbb{K}$ mit den Eigenschaften:
    \begin{enumerate}
      \item $det E = 1$
      \item Wenn $A$ zwei gleiche Zeilen besitzt, dann gilt $det A = 0$
      \item Die Funktion $det$ ist linear in jeder Zeile:
      \begin{enumerate}
        \item $det(z_1,\dots,\lambda z_i,\dots,z_n) = 
              \lambda det(z_1,\dots,z_i,\dots,z_n)$
        \item $det(z_1,\dots,z_i+z,\dots,z_n) = 
              det(z_1,\dots,z_i,\dots,z_n) + det(z_1,\dots,z,\dots,z_n)$
      \end{enumerate}
    \end{enumerate}

  \item [Lösungsmenge:]
    Die Lösungsmenge eines LGS ist die Menge aller Vektoren, die alle
    Gleichungen simultan erfüllen:
    \begin{equation*}
      \mathbb{L}_{A,\vec{b}} := 
      \{(u_<, \dots,u_n) \in \mathbb{R}^n | A \cdot \vec{u} = \vec{b}\}
    \end{equation*}

  \item [Rang:]
    Der Rang einer Matrix $A \in \mathbb{K}^{m \times n}$ ist das Bild
    ($A : \mathbb{K}^n \rightarrow \mathbb{K}^m$).

  \item [Zeilenrang:]
    Der Zeilenrang einer Matrix ist die Maximalzahl linear unabhängiger Zeilen.

  \item [Spaltenrang:]
    Der Spaltenrang einer Matrix ist die Maximalzahl linear unabhängiger Spalten.

  \item [linear unabhängig:]
    Die Vektoren $\vec{a_1},\dots,\vec{a_n}$ heißen linear unabhängig, wenn 
    $\lambda_1\vec{a_1} + \dots + \lambda_n\vec{a_n} = \vec{0}$ nur für die 
    triviale Lösung $\lambda_1 = \dots = \lambda_n = 0$ gilt.

  \item [Unterbestimmt:]
    Ein LGS ist unterbestimmt, wenn der Rang von $A$ kleiner als $n$ ist. D.h. 
    wenn es weniger Gleichungen als Variablen gibt ($m < n$).

  \item [Überbestimmt:]
    Ein LGS ist überbestimmt, wenn der Rang von $A$ größer als $n$ ist. D.h. 
    wenn es mehr Gleichungen als Variablen gibt ($m > n$).

  \item [Lineare Differentialgleichung:]
    Wir ersetzen die Potenz $x^n$ durch die $n$-fache Ableitung $f^{(n)}$ einer 
    Funktion $f$ und suchen eine Lösung für:
    \begin{equation*}
      a_nf^{(n)} + \dots a_1f' + a_0f = 0
    \end{equation*}

  \item [Erzeugendensystem:]
    Sei $X \subset V$. X ist ein Erzeugendensystem, wenn jeder beliebige Vektor 
    $v \in V$ durch eine Linearkombination
    $\lambda_1 \cdot v_1 + \dots + \lambda_n \cdot v_n$ darstellbar ist.

  \item [Basis:]
    Sei $V$ ein Vektorraum. Eine Menge $B \subset V$ heißt Basis, wenn $B$ 
    linear unabhängig ist und $V$ erzeugt. Dabei enthält eine Basis immer genau
    so viele Vektoren, wie der Vektorraum Dimensionen: $dim\mathbb{R}^n = n$
    Basisvektoren.

  \item [Kern:]
    Der Kern einer Matrix $A$ ist eine Menge von Vektoren, die durch
    Multiplikation mit einer Matrix $M$ den Nullvektor erzeugen. geschrieben:
    $Kern(M)$

  \item [Eindeutig lösbar:]
    Ein quadratisches LGS $A\vec{x} = \vec{b}$ ist genau dann eindeutig lösbar,
    wenn $detA \neq 0$.
\end{description}

\begin{framed} [Rechenregeln]
    \subsubsection*{Matrizen}
      \begin{center}
        $A = 
        \begin{pmatrix}
          a_{1,1} & \dots  & a_{1,n} \\
          \vdots  & \ddots & \vdots  \\
          a_{m,1} & \dots  & a_{m,n} \\
        \end{pmatrix}$
        wobei
        $A^T =
        \begin{pmatrix}
          a_{1,1} & \dots  & a_{m,1} \\
          \vdots  & \ddots & \vdots  \\
          a_{1,n} & \dots  & a_{m,n} \\
        \end{pmatrix}$
      \end{center}
    
    \subsubsection*{Addition:}
      \begin{center}
        $A + B =
        \begin{pmatrix}
          a_{1,1} + b_{1,1} & \dots  & a_{1,n} + b_{1,n} \\
          \vdots  	        & \ddots & 		 \vdots        \\
          a_{m,1} + b_{m,1} & \dots  & a_{m,n} + b_{m,n} \\
        \end{pmatrix}$
      \end{center}

    \subsubsection*{Multiplikation}
      \begin{center}
        $A \cdot B = 
        \begin{pmatrix}
          a_{1,1} & a_{1,2} \\
          a_{2,1} & a_{2,2} \\
          a_{3,1} & a_{3,2} \\
        \end{pmatrix}
        \cdot
        \begin{pmatrix}
          b_{1,1} \\
          b_{2,1} \\
        \end{pmatrix}
        =
        \begin{pmatrix}
          a_{1,1} \cdot b_{1,1} & a_{1,2} \cdot b_{2,1} \\
          a_{2,1} \cdot b_{1,1} & a_{2,2} \cdot b_{2,1} \\
          a_{3,1} \cdot b_{1,1} & a_{3,2} \cdot b_{2,1} \\
        \end{pmatrix}$
      \end{center}
 
    \subsubsection*{Einheitsmatrix}
      \begin{center}
        $\begin{pmatrix}
          1       & 0       & \dots   & 0       \\
          0       & 1       & \dots   & 0       \\
          \vdots  & \vdots  & \ddots  & \vdots  \\
          0       & 0       & \dots   & 1       \\ 
        \end{pmatrix}$
      \end{center}
      Die Einheitsmatrix ist das neutrale Element für Matrizen.\\
      Es gilt also: $A \cdot E = A$

    \subsubsection*{Berechnung der Determinanten:}
      \begin{center}
        für A =
        $\begin{pmatrix}
          a_{11}  & *       & \dots   & *       \\
          0       & a_{22}  & *       & \vdots  \\
          \vdots  & *       & a_{33}  & 0       \\
          0       & \dots   & 0       & a_{nn}  \\
        \end{pmatrix}$
        gilt $detA = a_{11} \dots a_{n,n}$
      \end{center}
      Dies geht nur bei Quadratischen Matrizen.
\end{framed}
