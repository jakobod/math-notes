\subsection*{Definitionen}
\begin{description}
  \item [Induktion:]
    Ist $A(n)$ eine von $n \in \mathbb{N}$ abhängige Aussage, so sind dazu die
    folgenden beiden Beweisschritte 1 und 3 durchzuführen:
    \begin{enumerate}
      \setlength{\itemsep}{-0.1cm}
      \item Induktionsanfang (IA): Man zeigt, dass $A(0)$ richtig ist.
      \item Induktionsbehauptung (IB): Annahme, dass $A(k)$ für ein festes $k$ 
            richtig ist.
      \item Induktionsschluss (IS): Man zeigt: Aus der Annahme, dass $A(k)$ 
            richtig ist (Induktionsanker), folgt, dass auch $A(k + 1)$ richtig 
            ist:\\
            $A(k) \Rightarrow A(k + 1)$
    \end{enumerate}
    Dann ist gewährleistet, dass $A(n)$ für alle $n \in \mathbb{N}$ gilt.\\
    \textbf{Wichtig:} Wird der Induktionsanfang nicht für $n_0 = 0$, sondern für
    ein $n_0 > 0$ durchgeführt, so gilt die Aussage nur für alle $n \geq n_0$.

  \item [Induktiv erzeugte Menge:]
    Die Menge $M$ wird wie folgt induktiv definiert:
    \begin{itemize}
      \setlength{\itemsep}{-0.1cm}
      \item Basismenge: jedes $x$ mit $\tilde{}$ gehört zu $M$.
      \item Erzeugungsregel: Sind $m_1$ und $m_2$ Elemente aus $M$, dann auch 
            das Element $m = randoomFunction(m_1,m_2)$.
      \item Nur Elemente, die so gebildet werden können, gehören zu $M$.
    \end{itemize}

  \item [Wörter/Zeichenketten:]
    Sei $\sum$ ein Alphabet. Die Menge $\sum^*$ aller Wörter über $\sum$ ist
    induktiv definiert:
    \begin{itemize}
      \setlength{\itemsep}{-0.1cm}
      \item Basismenge: Das leere Wort $\epsilon$ gehört zu $\sum^*$; das heißt:
            $\epsilon \in \sum^*$
      \item Erzeugungsregel: Ist $w$ ein Wort in $\sum^*$ und $a$ ein Element 
            von $\sum$, dann gehört die Konkatenation $wa$ zu $\sum^*$
    \end{itemize}

  \item [Länge eines Wortes:]
    Sei $\sum$ ein Alphabet. Die Länge eines Wortes $w \in \sum^*$ ist induktiv
    definiert durch:
    \begin{itemize}
      \setlength{\itemsep}{-0.1cm}
      \item Die Länge des leeren Wortes $\epsilon$ ist $0$, |$\epsilon| = 0$
      \item Sei $w \in \sum^*$ und $a \in \sum$. Dann ist $|wa| = |w| + 1$. 
    \end{itemize}

  \item [Aussagenlogische Formeln:]
    Sei $X$ die Menge der aussagenlogischen Variablen. Die Menge der 
    aussagenlogischen Formeln wird wie folgt definiert:
    \begin{itemize}
      \setlength{\itemsep}{-0.1cm}
      \item Basismenge: Die Konstanten $w$ und $f$ sind aussagenlogische Formeln.
      \item Basismenge: Jede Aussagenlogische Variable $x \in X$ ist eine 
            aussagenlogische Formel.
      \item Erzeugungsregel: Sind $\alpha$ und $\beta$ aussagenlogische Formeln, 
            so sind auch 
            $(\neg \alpha), (\alpha \wedge \beta), (\alpha \vee \beta), 
            (\alpha \Rightarrow \beta) und (\alpha \Leftrightarrow \beta)$ 
            aussagenlogische Formeln.
    \end{itemize}
\end{description}


\begin{framed} [Beispiel Induktion]
  \textbf{Satz:} Für alle natürlichen Zahlen $n$ gilt 
                 $\sum_{i=0}^{n} i = \frac{1}{2}n(n + 1)$.
  \begin{itemize}
    \setlength{\itemsep}{-0.1cm}
    \item Induktionsanfang (IA): Die Eigenschaft gilt für $n = 0$, denn 
          $\sum_{i=1}^{n} i = 0$ und $\frac{1}{2}n(n + 1) = 0$.
    \item Induktionsbehauptung (IB): Wir nehmen an, dass die Summenformel für 
          ein beliebiges, aber festes $k$ gilt: 
          $\sum_{i=0}^{k} i = \frac{1}{2}k(k + 1)$.
    \item Induktionsschluss (IS): Unter der Vorraussetzung, dass die IB gilt, 
          wollen wir die Summenformel für $k + 1$ zeigen: 
          $\sum_{i=1}^{k+1}i = \frac{1}{2}k(k + 1)(+2)$ Dies können wir durch 
          folgende Umformung zeigen:\\
          $\sum_{i=1}^{k+1}i = (\sum_{i=0}^{k})+(k+1) = \frac{k(k+1)}{2}+(k+1) =
          \frac{k(k+1)+(2k+2)}{2} = \frac{k^2+3k+2}{2} = \frac{(k+1)(k+2)}{2}$
  \end{itemize}
  Nach dem Induktionsprinzip gilt die Aussage also für alle natürlichen Zahlen.
\end{framed}

\begin{framed} [Verallgemeinerte vollständige Induktion]
  Diese wird genutzt, wenn die Induktionsannahme $A(n)$ nicht genug ist, um den
  Induktionsschluss $A(n+1)$ beweisen zu können.\\
  Die Aussage $A(n)$ gilt für alle natürlichen Zahlen, wenn sowohl der
  Induktionsanfang $A(0)$ als auch der Induktionsschritt gilt:
  \begin{equation*}
    \forall n \in \mathbb{N} : (A(0) \wedge \dots \wedge A(n)) \Rightarrow A(n+1)
  \end{equation*}

  \subsubsection*{Beispiel:} Sei $n$ eine natürliche Zahl und $n \geq 2$. Dann 
  ist $n$ das Produkt von Primzahlen. \\
  \begin{itemize}
    \setlength{\itemsep}{-0.1cm}
    \item \textbf{IA:} Die Eigenschaft gilt für $n = 2$, denn $2$ ist das 
          Produkt von sich selbst, also einer Primzahl.
    \item \textbf{Starke Induktionsbehauptung (IB):} für festes $n$ nehmen wir 
          an, dass sich alle Zahlen $2,3,\dots, n$ als Produkt von Primzahlen
          schreiben lassen.
    \item \textbf{Induktionsschluss (IS):} Z.z.: $n + 1$ ist ein Produkt von 
          Primzahlen.\\
          Fall 1: $n + 1$ ist eine Primzahl, dann auch ein Produkt von Primzahlen
          (sich selbst).\\
          Fall 2: $n + 1$ ist keine Primzahl, dann gibt es mindestens zwei echte
          Teiler $b$ und $c$:\\
          Da $b$ und $c$ beide echt kleiner als $n + 1$ sind, gilt die IB für sie:
          \begin{equation*}
            b = p_1\dots p_k \space\space c = q_1\dots q_l
          \end{equation*}
          Damit gilt: $n + 1 = b \cdot c = (p_1\dots p_k)\cdot(q_1\dots q_l)$\\
          $n + 1$ ist also ein Produkt von Primzahlen.
  \end{itemize}
  Nach dem verallgemeinerten Induktionsprinzip gilt die Aussage also für alle 
  natürlichen Zahlen $n \geq 2$.
\end{framed}

\begin{framed}[\dq{}Induktionsschablone\dq{}]
  \begin{enumerate}
    \setlength{\itemsep}{-0.1cm}
    \item \textbf{(IA):} Die Eigenschaft gilt für $n = 0$, denn $\dots$
    \item \textbf{(IB):} Wir nehmen an, dass die Summenformel für ein beliebiges,
          aber festes $k$ gilt: $\sum_{i=0}^{k} \dots$
    \item \textbf{(IS):} Unter der Voraussetzung, dass die IB gilt, wollen wir 
          die Summenformel für $k + 1$ zeigen: $\sum_{i=0}^{k+1} \dots$
  \end{enumerate}
  Nach dem Induktionsprinzip gilt die Aussage also für alle natürlichen Zahlen.
\end{framed}

\begin{framed} [Peano-Axiome:]
  \begin{equation*}
    \begin{aligned}
      0 \cdot m &= 0\\
      (n + 1) \cdot m &= (n \cdot m) + m\\
      n \cdot (m_1 + m_2) &= n \cdot m_1 + n \cdot m_2
    \end{aligned}
  \end{equation*}
\end{framed}
