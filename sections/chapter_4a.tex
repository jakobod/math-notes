\subsection*{Definitionen}
\begin{description}
  \item [Term:] 
    Ein Term setzt sich zusammen aus:
    \begin{itemize}
      \item Konstanten: $e, \pi, \dots$
      \item Variablen: $x, y, \dots$
      \item Operatoren: $+, -, \times, \sqrt{}, \dots$ 
      \item Funktionen: $f(x), sin(x), \dots$
    \end{itemize}
    
  \item [Gleichung:] 
    Eine Gleichung $t_1 = t_2$ setzt zwei Terme in Beziehung.

  \item [Grad (von Fkt.):]
    Ein reellwertiges Polynom ist ein Ausdruck der Form ($x \in \mathbb{R}$):\\
    $p(x) = a_nx^n + a_{n-1}x^{n-1} + \dots + a_nx^n + a_1x^1 + a_0$\\
    Falls $a_n \neq 0$, dann ist $n$ der Grad des Polynoms.

  \item [Logarithmus:]
    $log_a(c) = x$ für $a^x = c$
\end{description}

\begin{framed} [Brüche]
  \begin{tabular} {l|l}
    \thead[lc]{Addition bei\\
              gleichem Nenner:} & 
    \parbox{3cm}{\begin{equation*}
      \begin{aligned}
        \frac{a}{d} + \frac{b}{d} = \frac{a + b}{d}
      \end{aligned}
    \end{equation*}}
    \\
    \thead[lc]{Multiplikation:} & 
    \parbox{3cm}{\begin{equation*}
      \begin{aligned}
        \frac{a}{c} + \frac{b}{d} = \frac{ab}{cd}
      \end{aligned}
    \end{equation*}}
    \\
    \thead[lc]{Kürzen eines\\
              gleichen Faktors:} &
    \parbox{3.5cm}{\begin{equation*}
      \begin{aligned}
        \frac{a \cdot c}{d \cdot c} = \frac{a}{d} \cdot \frac{c}{c} = \frac{a}{d} 
        \cdot 1 = \frac{a}{d}
      \end{aligned}
    \end{equation*}}
    \\
    \thead[lc]{Erweitern um c:} &
    \parbox{3.5cm}{\begin{equation*}
      \begin{aligned}
        \frac{a}{d} = \frac{a}{d} \cdot 1 = \frac{a}{d} \cdot \frac{c}{c} = 
        \frac{a \cdot c}{d \cdot c}\notag
      \end{aligned}
    \end{equation*}}
    \\
    \thead[lc]{Addieren mit\\
              verschiedenen Nennern\\
              (durch erweitern):} &
    \parbox{5.3cm}{\begin{equation*}
      \begin{aligned}
        \frac{a}{c} + \frac{b}{d} = (\frac{a}{c} \cdot \frac{d}{d}) + 
        (\frac{b}{d} \cdot \frac{c}{c}) = \frac{ad + bc}{cd}\notag
      \end{aligned}
    \end{equation*}}
  \end{tabular}
\end{framed}
  
\subsection*{Summen-/Produktnotation}
\begin{description}
  \item [Summennotation:]
    $\sum_{i=1}^{n}a_i := a_1 + a_2 + \dots + a_n$ ist äquivalent zu 
    $\sum_{1\leq i\leq n}^{}a_i$

  \item [Produktnotation:]
    $\prod_{i=1}^{n}a_i := a_1 \cdot a_2 \cdot \dots \cdot a_n$
\end{description}

\begin{framed} [Exponential-Gesetze]
  Sei $m,n \in \mathbb{N}:$
  \begin{itemize}
    \item $b^{-n} = \frac{1}{b^n}$
    \item $b^m \cdot b^n = b^{m+n}$
    \item $\frac{b^m}{b^n} = b^{m-n}$
    \item $(b^m)^n = b^{m\cdot n}$
    \item $b^m \cdot c^m = (b\cdot c)^m$
    \item $\frac{b^m}{c^m} = (\frac{b}{c})^m$
    \item $\sqrt[n]{x^m} = x^\frac{m}{n}$
    \item $x^{-\frac{m}{n}} = \frac{1}{\sqrt[n]{x^m}}$
  \end{itemize}
\end{framed}

\begin{framed} [Binomische Formeln]
  \begin{itemize}
    \item $(a + b)^2 = a^2 + 2ab + b^2$
    \item $(a - b)^2 = a^2 - 2ab + b^2$
    \item $(a - b) \cdot (a + b) = a^2 - b^2$
  \end{itemize}		
\end{framed}
  

\begin{framed} [Logarithmen]
  \begin{tabular} {l|l}
    \thead[lc]{Umkehrung:} & 
    \parbox{4cm}{\begin{equation*}
      \begin{aligned}
        log_b(b^q) = q \Leftrightarrow b^{log_b(q)} = q
      \end{aligned}
    \end{equation*}}
    \\
    \thead[lc]{Mul/Additivität:} & 
    \parbox{4.2cm}{\begin{equation*}
      \begin{aligned}
        log_b(xy) &= log_b(x) + log_b(y)\\
        log_b(a^q) &= q log_b(a)
      \end{aligned}
    \end{equation*}}
    \\
    \thead[lc]{Basiswechsel:} & 
    \parbox{3cm}{\begin{equation*}
      \begin{aligned}
        log_b(a) = \frac{log_d(a)}{log_d(b)}
      \end{aligned}
    \end{equation*}}
  \end{tabular}
\end{framed}


\begin{framed} [Trigonometrische Funktionen]
  \begin{tabular} {l|l}
    \thead{Tangensfunktion:} & 
    \parbox{2.5cm}{\begin{equation*}
      \begin{aligned}
        tan(x) := \frac{sin(x)}{cos(x)}
      \end{aligned}
    \end{equation*}}
    \\
    \thead{Co-Tangensfunktion:} & 
    \parbox{2.5cm}{\begin{equation*}
      \begin{aligned}
        cot(x) := \frac{cos(x)}{sin(x)}
      \end{aligned}
    \end{equation*}}
  \end{tabular}
\end{framed}


\begin{framed} [Zusammenfassung]
  \begin{itemize}
    \item
      Ein Term ist ein Ausdruck, der für einen Zahlenwert steht. Ein Beispiel für 
      einen Term ist z.B: $sin(x)^2 + cos(x^2)$\\
      Wobei $x^2 - 3 = 0$ kein Term ist, sondern eine Gleichung.

    \item
      für die Summen-/Produktnotation gibt man unten die untere Grenze und oben 
      die obere Grenze die Summierung/Produktbildung an. Bsp:\\
      $\sum_{i=1}^{5}a_i := 1 + 2 + 3 + 4 + 5$\\
      $\prod_{0\leq i\leq 5, i\%2=1}^{} (i + 1):= (1 + 1) \cdot (3 + 1) 
      \cdot (5 + 1)$

    \item
      Hyperbeln können nach ähnlichem Schema umgeformt werden. Dabei wird ein Term 
      $\frac{1}{x}$ zu $x^{-1}$. Wenn der Nenner einen Exponenten besitzt kann 
      dieser auch wieder umgeformt werden, damit die Rechengesetze gelten.	
      $\frac{1}{x^m} = x^{-m}$

    \item
      Wurzelfunktionen können umgeformt werden. Wenn man z.B einen Term
      $\sqrt[2]{x}$ hat, kann dieser zu $x^\frac{1}{2}$ umgeformt werden. Dadurch
      gelten auch für Wurzeln, die Exponentialgesetze.	

    \item
      Logarithmen können genutzt werden, wenn bei einer Exponentialfkt. der 
      Exponent unbekannt, aber das Ergebnis bekannt ist. Dabei gibt es den 
      Spezialfall des natürlichen Logarithmus. Dieser ist definiert als:\\
      $ln(x) := log_e(x)$
  \end{itemize}
\end{framed}


%		\section*{Rechenregeln für reelle Zahlen}
%		\begin{equation}
%		\textbf{Assoziativität: }
%		\begin{aligned}
%		(x + y) + z = x + (y + z)\notag
%		\end{aligned}
%		\end{equation}
%		
%		\begin{equation}
%		\textbf{Neutralität der 0: }
%		\begin{aligned}
%		x + 0 = x\notag
%		\end{aligned}
%		\end{equation}
%		
%		\begin{equation}
%		\textbf{Kommutativität: }
%		\begin{aligned}
%		(x + y) + z = x + (y + z)\notag
%		\end{aligned}
%		\end{equation}