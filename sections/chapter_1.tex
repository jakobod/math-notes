\subsection*{Definitionen}
\vspace{-.4cm}
\begin{description}
  \setlength{\itemsep}{-.2cm}
  \item [Aussage:] 
    Eine Aussage ist ein Satz, der entweder wahr ($w$) oder falsch ($f$) ist, 
    also nie beides zugleich.
  
  \item [Konjunktion:] 
    Seien $A$ und $B$ Aussagen. Dann ist auch $A$ \textbf{und} $B$ eine Aussage
    - die sogenannte \textbf{(logische) Konjunktion}. Kurz: ($A \wedge B$). \\
    ($A \wedge B$) ist genau dann wahr, wenn sowohl $A$ als auch $B$ wahr ist.
  
  \item [Disjunktion:] 
    Seien $A$ und $B$ Aussagen. Dann ist auch $A$ \textbf{oder} $B$ eine Aussage
    - die sogenannte \textbf{(logische) Disjunktion}. Kurz: ($A \vee B$). \\
    ($A \vee B$) ist genau dann wahr, wenn $A$ oder $B$ wahr ist --- oder beide.
  
  \item [Negation:] 
    Sei $A$ eine Aussage. Dann ist auch \textbf{nicht} $A$ eine Aussage - die 
    \textbf{(logische) Negation}. Kurz: ($\neg A$). \\
    Die Aussage ($\neg A$) ist genau dann wahr, wenn $A$ falsch ist.
  
  \item [Implikation:] 
    Seien $A$ und $B$ Aussagen. Dann ist auch \textbf{wenn} $A$, \textbf{dann} 
    $B$ eine Aussage, die \textbf{(logische) Implikation}.
    Kurz: ($A \Rightarrow B$). \\
    Die Aussage ($A \Rightarrow B$) ist genau dann falsch, wenn $A$ wahr und $B$ 
    falsch ist.
  
  \item [Bi-Implikation:] 
    Seien $A$ und $B$ Aussagen. Dann ist auch $A$ \textbf{genau dann, wenn} 
    $B$ eine Aussage, die \textbf{Bi-Implikation}. Kurz: (A $\iff$ B). Die 
    Aussage (A $\iff$ B) ist genau dann wahr, wenn A und B beide den gleichen 
    Wahrheitswert haben.
  
  \item [Formel:] 
    Eine aussagenlogische Verknüpfungvon Aussagenvariablen (durch endlich viele 
    Junktoren) heißt \textbf{(aussagenlogische) Formel}. Aussagenvariablen 
    werden auch als \textbf{atomare Formeln} bezeichnet.

  \item [Auswertung $A_{B}(x)$:] 
    Sei $B : Var \to \{w,f\}$ eine Belegung. Die Auswertung $A_{B}(F)$ 
    ergibt sich \textbf{rekursiv}; das Rekursionsende sind die Variablen:
  
  \item [Tautologie:] 
    Eine \textbf{Tautologie} ist eine Formel, die stets wahr is, in deren 
    Wahrheitswerteverlauf also ausschließlich den Wahrheitswert $w$ 
    vorkommt.
  
  \item [Kontradiktion:] 
    Eine \textbf{Kontradiktion} ist eine Formel, die stets falsch ist, in deren 
    Wahrheitswerteverlauf also ausschließlich der Wahrheitswert $f$ vorkommt.
  
  \item [Erfüllbarkeit:] 
    Eine Formel $F$ heißt \textbf{erfüllbar}, wenn es mindestens eine 
    Belegung der Aussagenvariablen gibt, die $F$ wahr macht.
  
  \item [Äquivalenz:] 
    Zwei Formeln F und G heißen \textbf{(logisch) äquivalent} genau dann, wenn 
    die Formel ($F \iff G$) eine Tautologie ist. Dies wird durch 
    $F \equiv G$ dargestellt.
  
  \item [Folgerungsbeziehung:] 
    Die Formel $F$ \textbf{imliziert die Formel} $G$ genau dann, 
    wenn ($F \Rightarrow G$) eine Tautologie ist. \\
    Dies wird durch $F |= G$ dargestellt. (Aus $F$ folgt $G$)
  
  \item [Aussageform:] 
    Eine \textbf{Aussageform} Über den Universen $U_{1}$, $\dots$, $U_{n}$ 
    ist ein Satz mit den freien Variablen $x_{1}$, $\dots$, $x_{n}$.

  \item [Quantoren:] 
    Sei $p(x)$ eine Aussageform über dem Universum U. 
    $\exists x : p(x)$ ist wahr genau dann, wenn ein $u$ in 
    $U$ existiert, so dass	$p(u)$ wahr ist. \\
    $\forall : p(x)$ ist wahr genau dann, wenn $p(u)$ für jedes 
    $u$ aus $U$ wahr ist.
  
  \item [gebundene Variablen:] 
    Eine Variable $x$ wird in einer Formel 
    $F = \forall x : G$ durch den Allquantor 
    \textbf{gebunden}.\\
    Analog wird $x$ in $F = \exists x : G$ 
    durch den Existenzquantor \textbf{gebunden}.
  
  \item [Normalisierte Darstellung:] 
    Eine Formel ist in normalisierter Variablenschreibweise, wenn gilt:
    \vspace{-.5cm}
    \begin{enumerate}
      \setlength{\itemsep}{-0cm}
      \item keine Variable kommt sowohl frei, als auch gebunden vor
      \item keine Variable ist mehrfach gebunden
    \end{enumerate}
\end{description}


\subsection*{Wahrheitstabellen}
\begin{tabular}{l l l}
  \begin{tabular}{ | c | c | c | }
    \hline
    \textbf{A} & \textbf{B} & \textbf{(A $\wedge$ B)} \\ \hline
    w & w & w \\ \hline
    w & f & f \\ \hline
    f & w & f \\ \hline
    f & f & f \\ \hline
  \end{tabular}
  &
  \begin{tabular}{ | c | c | c | }
    \hline
    \textbf{A} & \textbf{B} & \textbf{(A $\vee$ B)} \\ \hline
    w & w & w \\ \hline
    w & f & w \\ \hline
    f & w & w \\ \hline
    f & f & f \\ \hline
  \end{tabular}
  &
  \begin{tabular}{ | c | c | }
    \hline
    \textbf{A} & \textbf{$\neg$A} \\ \hline
    w & f \\ \hline
    f & w \\ \hline
  \end{tabular}
  \\
  \begin{tabular}{ | c | c | c | }
    \hline
    \textbf{A} & \textbf{B} & \textbf{(A $\Rightarrow$ B)} \\ \hline
    w & w & w \\ \hline
    w & f & f \\ \hline
    f & w & w \\ \hline
    f & f & w \\ \hline
  \end{tabular}
  &
  \begin{tabular}{ | c | c | c | }
    \hline
    \textbf{A} & \textbf{B} & \textbf{(A $\iff$ B)} \\ \hline
    w & w & w \\ \hline
    w & f & f \\ \hline
    f & w & f \\ \hline
    f & f & w \\ \hline
  \end{tabular}
  &
\end{tabular}	

\vspace{-.3cm}

\subsection*{Umformungsregeln}
\begin{framed} [Kommutativgesetz] 
  \begin{equation*}
    \begin{aligned}
      (p \wedge q) \equiv (q \wedge p) \\
      (p \vee q) \equiv (q \vee p) 
    \end{aligned}
  \end{equation*}
\end{framed}

\vspace{-.7cm}

\begin{framed} [Assoziativgesetz] 
  \begin{equation*}
    \begin{aligned}
      (p \wedge (q \wedge r)) \equiv ((p \wedge q) \wedge r) \\
      (p \vee (q \vee r)) \equiv ((p \vee q) \vee r) 
    \end{aligned}
  \end{equation*}
\end{framed}

\vspace{-.7cm}

\begin{framed} [Distributivgesetz] 
  \begin{equation*}
    \begin{aligned}
      (p \wedge (q \vee r)) \equiv ((p \wedge q) \vee (p \wedge r)) \\	(p \vee (q \wedge r)) \equiv ((p \vee q) \wedge (p \vee r)) 
    \end{aligned}
  \end{equation*}
\end{framed}

\vspace{-.7cm}

\begin{framed} [Idempotenzgesetz] 
  \begin{equation*}
    \begin{aligned}
      (p \wedge p) \equiv p \\
      (p \vee p) \equiv p 
    \end{aligned}
  \end{equation*}
\end{framed}

\vspace{-.7cm}

\begin{framed} [Doppelnegation] 
  \begin{equation*}
    \begin{aligned}
      \neg(\neg p) \equiv p 
    \end{aligned}
  \end{equation*}
\end{framed}

\vspace{-.7cm}

\begin{framed} [de Morgan Gesetz] 
  \begin{equation*}
    \begin{aligned}
      \neg(p \wedge q) \equiv ((\neg p)\vee (\neg q)) \\ 
      \neg(p \vee q) \equiv ((\neg p)\wedge (\neg q)) 
    \end{aligned}
  \end{equation*}	
\end{framed}

\vspace{-.7cm}

\begin{framed} [Tautologieregeln] 
  \begin{center}
    \vspace{-.3cm}
    \begin{tabular} {c c}
      \parbox{.5\textwidth}{\begin{equation*}
        wobei: (q = Tautologie)
      \end{equation*}}
      &
      \parbox{3cm}{\begin{equation*}
        \begin{aligned}
          (p \wedge q) \equiv p \\ 
          (p \vee q) \equiv q
        \end{aligned}
      \end{equation*}}
    \end{tabular}
    \vspace{-.3cm}
  \end{center}
\end{framed}

\vspace{-.7cm}

\begin{framed} [Kontradiktionsregeln] 
  \begin{center}
    \vspace{-.3cm}
    \begin{tabular} {c c}
      \parbox{.5\textwidth}{\begin{equation*}
        wbei: (q = Kontradiktion) 
      \end{equation*}}
      &
      \parbox{3cm}{\begin{equation*}
        \begin{aligned}
          (p \wedge q) \equiv q \\ 
          (p \vee q) \equiv p 
        \end{aligned}
      \end{equation*}}
    \end{tabular}
    \vspace{-.3cm}
  \end{center}
\end{framed}
\vspace{-.7cm}

\subsection*{Umformungsregeln für Quantoren:}
\begin{framed} [Negationsregeln]
  \begin{equation*}
    \begin{aligned}
      \neg \forall x : p(x) \equiv \exists x : (\neg p(x)) \\
      \neg \exists x : p(x) \equiv \forall x : (\neg p(x))	
    \end{aligned}
  \end{equation*}
\end{framed}

\vspace{-.7cm}

\begin{framed} [Ausklammerregeln]
  \begin{equation*}
    \begin{aligned}
      (\forall x : p(x) \wedge \forall y : q(y)) \equiv \forall z : (p(z) \wedge q(z)) \\
      (\exists x : p(x) \vee \exists y : q(y)) \equiv \exists z : (p(z) \vee q(z))
    \end{aligned}
  \end{equation*}
\end{framed}

\vspace{-.7cm}

\begin{framed} [Vertauschungsregeln]
  \begin{equation*}
    \begin{aligned}
      \forall x \forall y : p(x,y) \equiv \forall y \forall x : p(x,y)\\
      \exists x \exists y : p(x,y) \equiv \exists y \exists x : p(x,y)
    \end{aligned}
  \end{equation*}
\end{framed}


