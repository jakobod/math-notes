%\small %oder \tiny oder \scriptsize oder \footnotesize oder \small ...
\section*{Definitionen}
\textbf{1. Permutation:} Eine Anordnung aller Elemente einer endlichen Menge hei�t Permutation.\\
\textbf{Ausf�hrlicher:} Eine bijektive Abbildung einer endlichen Menge $M$ auf sich selbst nennt man Permutation der Menge.\\
Es gibt also $|M|!$ Permutationen von $M$.\\
Die Menge aller Permutationen der Menge $\{1,\dots,n\}$ bezeichnen eir mit $S_n$. Es gilt somit $|S_n| = n!$.

\textbf{2. $k$-Permutation:} Eine $k$-Permutation einer endlichen Menge $S$ ist eine Permutation einer $k$-elementigen Teilmenge von $S$.\\
Die Anzahl aller $k$-Permutationen einer $n$-elementigen Menge wird mit $[\frac{n}{k}]$ bezeichnet. (auch: $(n)_k$)\\
Die Anzahl der $k$-elementigen Teilmenge einer $n$-elementigen Menge wird mit $(\frac{n}{k})$ bezeichnet. Dies ist der Binomialkoeffizient gesprochen \glqq $n$ �ber $k$\grqq\\
$(\frac{n}{k})$ kann auch als $\frac{n!}{k! \cdot (n - k)}$ dargestellt werden.\\
\textbf{Beispiel:}\\
Sei $S = \{1,2,3\}$ \\
Die 2-Permutation von $S$ ist also:\\
$(1,2), (2,1), (1,3), (3,1), (2,3), (3,2)$\\
Der Binomialkoeffizient hier ist: $(\frac{n}{k}) = (\frac{3}{2}) = 3$

\textbf{3. Begriffserkl�rungen Wahrscheinlichkeitstheorie:}
\begin{itemize}
  \item Der Ereignisraum $\Omega := \{\omega_1,\omega_2,\dots,\omega_n\}$. Hier: endlich und diskret.
  \item Die Menge der Ereignisse $\mathcal{A} = \mathcal{P}(\Omega)$, also die Potenzmenge von $\Omega$.
  \item Das Wahrscheinlichkeitsma� $P : \mathcal{P}(\Omega) \rightarrow \mathbb{R}$ ordnet jedem Ereignis $A \in \mathcal{P}(\Omega)$ seine Wahrscheinlichkeit $P(A)$ zu.
  \item Ein Laplace-Versuch ist ein Zufallsversuch mit endlich vielen und gleich wahrscheinlichen Ergebnissen. Beispiel: M�nzwurf, W�rfel
  \item Bei Laplace-Versuchen wird das Wahrscheinlichkeitsma� durch eine Abz�hlregel definiert (Ereignis $A \in \mathcal{P}(\Omega)$):
  \begin{center}
    $P(A) = \frac{|A|}{|\Omega|}$
  \end{center}
  Also: P(A) ist das Verh�ltnis von der Anzahl der (f�r $A$) g�nstigen F�lle und der Anzahl aller insgesamt m�glichen F�lle.
  \item Man ben�tigt kombinatorische Prinzipien, um |A| zu bestimmen.
\end{itemize}

\textbf{Pascal'sche Gleichung:} $(\frac{n}{k}) = (\frac{n-1}{k-1})+(\frac{n-1}{k})$

%	\section*{Zusammenfassung}
%	Um die Anzahl von Kombinationen von endlichen, verschiedenen Objekten zu berechnen, wird die Kombinatorik gebraucht. Dies wir z.B. in der \glqq Diskreten Wahrscheinlichkeitstheorie\grqq gebraucht.

% Define block styles
\tikzstyle{decision} = [diamond, draw, fill=blue!20,
text width=4.5em, text centered, node distance=2.5cm, inner sep=0pt]
\tikzstyle{block} = [rectangle, draw, fill=blue!20,
text width=5em, text centered, rounded corners, minimum height=4em]
\tikzstyle{line} = [draw, very thick, color=black!50, -latex']
\tikzstyle{cloud} = [draw, ellipse,fill=red!20, node distance=2.5cm,
minimum height=2em]


\begin{tikzpicture}[node distance = 2cm, auto]

  % Place nodes
  \node [decision] (auswahl) {Auswahl?};
  \node [decision, below of=auswahl, left of=auswahl, xshift=-2cm] (wiederholung1) {Wdh?};
  \node [block, below of=wiederholung1, left of=wiederholung1] (blck1) {n!};
  \node [block, below of=wiederholung1, right of=wiederholung1] (blck2) {$\frac{k!}{m_1!\cdot m_2! \dots m_n!}$};
  
  \node [decision, right of=auswahl, below of=auswahl, xshift=2cm] (reihenfolge2) {Reihenfolge?};
  \node [decision, below of=reihenfolge2, left of=reihenfolge2, xshift=-0.7cm] (wdh2) {Wdh?};
  \node [block, below of=wdh2, left of=wdh2] (blck3) {$n^k$};
  \node [block, below of=wdh2, right of=wdh2] (blck4) {$\frac{n!}{(n-k)!}$};
  
  \node [decision, below of=reihenfolge2, right of=reihenfolge2, xshift=1cm] (wdh3) {Wdh?};
  \node [block, below of=wdh3, left of=wdh3] (blck5) {$\binom{n+k-1}{k}$};
  \node [block, below of=wdh3, right of=wdh3] (blck6) {$\binom{n}{k}$};
  
  
  
  % lines
  \path [line] (auswahl) -| (wiederholung1) node[draw=none, pos=0.1, left=4pt, label=NEIN]{};
  \path [line] (auswahl) -| (reihenfolge2) node[draw=none, pos=0.3, left=4pt, label=JA]{};
  
  \path [line] (wiederholung1) -| (blck1) node[draw=none, pos=0.1, left=4pt, label=NEIN]{};
  \path [line] (wiederholung1) -| (blck2) node[draw=none, pos=0.3, left=4pt, label=JA]{};
  
  \path [line] (reihenfolge2) -| (wdh2) node[draw=none, pos=0.1, left=4pt, label=JA]{};
  \path [line] (reihenfolge2) -| (wdh3) node[draw=none, pos=0.3, left=4pt, label=NEIN]{};
  
  \path [line] (wdh2) -| (blck3) node[draw=none, pos=0.1, left=4pt, label=JA]{};
  \path [line] (wdh2) -| (blck4) node[draw=none, pos=0.3, left=4pt, label=NEIN]{};
  
  \path [line] (wdh3) -| (blck5) node[draw=none, pos=0.1, left=4pt, label=JA]{};
  \path [line] (wdh3) -| (blck6) node[draw=none, pos=0.3, left=4pt, label=NEIN]{};


\end{tikzpicture}
