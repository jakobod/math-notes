
\section*{Definitionen}
\textbf{1. Gruppe:} Eine Menge ($\mathbb{G},\cdot, 1$) mit der Operation $\cdot$ und dem Element 1 ist eine Gruppe, wenn gilt:
\begin{itemize}
  \setlength{\itemsep}{-0.4cm}
  \item $(x \cdot y) \cdot z = x \cdot (y \cdot z)$ \textbf{Assoziativit�t}\\
  \item $x \cdot 1 = x$ \textbf{Neutralit�t der 1}\\
  \item $x \cdot x^{-1} = 1$ \textbf{Existenz von inversen $x^{-1}$ f�r $x \neq 0$}
\end{itemize}
Dabei hei�t eine Gruppe kommutativ, wenn zus�tzlich gilt:\\
$x \cdot y = y \cdot x$ \textbf{Kommutativit�t}

\textbf{2. Halbgruppe:} Eine Halbgruppe ist eine Verallgemeinerung einer Gruppe, der die Assoziativit�t gen�gt.

\textbf{3. K�rper:} Ein K�rper ($\mathbb{K}, +, 0, \cdot, 1$) ist eine Menge $\mathbb{K}$, welche mit zwei zweistelligen Verkn�pfungen versehen ist und folgende Eigenschaften besitzt:
\begin{itemize}
  \item ($\mathbb{K}, +, 0$) ist eine kommutative Gruppe, wobei 0 das neutrale Element der Addition ist.
  \item ($\mathbb{K}\backslash \{0\}, \cdot, 1$) ist eine kommutative Gruppe, wobei 1 das neutrale Element der Multiplikation ist.
  \item Des weiteren gilt das Distributivgesetz:
  $x(y + z) = xy + xz$
\end{itemize}

\textbf{4. Ring:} Ein Ring ($\mathbb{P}, +, 0, \cdot, 1$) besitzt folgende Eigenschaften:
\begin{itemize}
  \item ($\mathbb{P}, +, 0$) ist eine Gruppe.
  \item ($\mathbb{P}\backslash\{0\}, \cdot, 1$) ist eine Halbgruppe.
  \item Es gelten die Distributivgesetze.
\end{itemize}
Ein Ring hei�t kommutativ, wenn die Addition kommutativ ist. ($\mathbb{P}, +, 0$) also eine kommutative Gruppe ist.

\textbf{5. Unit�rer Ring:} Ein unit�rer Ring besitzt ein multiplikativ neutrales Element. ($\rightarrow$ 1)

\textbf{6. komplexe Zahl:} Eine Komplexe Zahl $z$ ist definiert als $z = a + b\cdot i$ mit $a,b \in \mathbb{R}$.\\
Dabei ist $a$ der Realteil und $b$ der Imagin�rteil von $z$.

\textbf{7. Exponentialdarstellung:} Eine komplexe Zahl l�sst sich auch mit Hilfe der komplexen $e$-Funktion darstellen:\\
$z = r \cdot e^{i\phi} = r \cdot (cos(\phi) + i \cdot sin(\phi))$\\
Dabei ist \\
$sin(\phi) = Im(z)$\\
$cos(\phi) = Re(z)$

\section*{Rechengesetze Imagin�re Zahlen}
\textbf{Addition:} $(a + bi) + (c + di) = (a + c) + (b + d)i$\\\\
\textbf{Subtraktion:} $(a + bi) - (c + di) = (a - c) + (b - d)i$\\\\
\textbf{Betrag Exponentialform:} $r = \sqrt{|a|^2 + |b|^2}$\\\\
\textbf{Winkel Exponentialform:} $\phi = arctan(\frac{Im}{Re})$\\\\
\textbf{Multiplikation:} $(r_1 \cdot e^{i\phi_1}) \cdot (r_2 \cdot e^{i\phi_2}) = r_1 \cdot r_2 \cdot e^{i(\phi_1+\phi_2)}$\\\\
\textbf{Disvision:} $\frac{r_1 \cdot e^{i\phi_1}}{r_2 \cdot e^{i\phi_2}} = \frac{r_1}{r_2} \cdot e^{i(\phi_1-\phi_2)}$
