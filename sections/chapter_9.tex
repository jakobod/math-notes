
\subsection*{Definitionen}
\begin{description}  
  \item [Gruppe:] 
    Eine Menge ($\mathbb{G},\cdot, 1$) mit der Operation $\cdot$ und dem Element
    1 ist eine Gruppe, wenn gilt:
    \begin{itemize}
      \item $(x \cdot y) \cdot z = x \cdot (y \cdot z)$ \textbf{Assoziativität}
      \item $x \cdot 1 = x$ \textbf{Neutralität der 1}
      \item $x \cdot x^{-1} = 1$ \textbf{Existenz von inversen $x^{-1}$ für 
            $x \neq 0$}
    \end{itemize}
    Dabei heißt eine Gruppe kommutativ, wenn zusätzlich gilt:\\
    $x \cdot y = y \cdot x$ \textbf{Kommutativität}

  \item [Halbgruppe:]
    Eine Halbgruppe ist eine Verallgemeinerung einer Gruppe, der die 
    Assoziativität genügt.

  \item [Körper:]
    Ein Körper ($\mathbb{K}, +, 0, \cdot, 1$) ist eine Menge $\mathbb{K}$, 
    welche mit zwei zweistelligen Verknüpfungen versehen ist und folgende 
    Eigenschaften besitzt:
    \begin{itemize}
      \item ($\mathbb{K}, +, 0$) ist eine kommutative Gruppe, wobei 0 das
            neutrale Element der Addition ist.
      \item ($\mathbb{K}\backslash \{0\}, \cdot, 1$) ist eine kommutative 
            Gruppe, wobei 1 das neutrale Element der Multiplikation ist.
      \item Des weiteren gilt das Distributivgesetz:
            $x(y + z) = xy + xz$
    \end{itemize}

  \item [Ring:]
    Ein Ring ($\mathbb{P}, +, 0, \cdot, 1$) besitzt folgende Eigenschaften:
    \begin{itemize}
      \item ($\mathbb{P}, +, 0$) ist eine Gruppe.
      \item ($\mathbb{P}\backslash\{0\}, \cdot, 1$) ist eine Halbgruppe.
      \item Es gelten die Distributivgesetze.
    \end{itemize}
    Ein Ring heißt kommutativ, wenn die Addition kommutativ ist. 
    ($\mathbb{P}, +, 0$) also eine kommutative Gruppe ist.

  \item [Unitärer Ring:]
    Ein unitärer Ring besitzt ein multiplikativ neutrales Element. 
    ($\rightarrow$ 1)

  \item [Komplexe Zahl:] 
    Eine Komplexe Zahl $z$ ist definiert als $z = a + b\cdot i$ mit 
    $a,b \in \mathbb{R}$.\\
    Dabei ist $a$ der Realteil und $b$ der Imaginärteil von $z$.

  \item [Exponentialdarstellung:]
    Eine komplexe Zahl lässt sich auch mit Hilfe der komplexen $e$-Funktion 
    darstellen:
    \begin{equation*}
      z = r \cdot e^{i\phi} = r \cdot (cos(\phi) + i \cdot sin(\phi))
    \end{equation*}
    Dabei ist
    \begin{equation*}
      \begin{aligned}
        sin(\phi) &= Im(z)\\
        cos(\phi) &= Re(z)
      \end{aligned}
    \end{equation*}
\end{description}

\begin{framed} [Rechengesetze Imaginäre Zahlen]
  \begin{description}
    \item [Addition:]
      $(a + bi) + (c + di) = (a + c) + (b + d)i$

    \item [Subtraktion:]
      $(a + bi) - (c + di) = (a - c) + (b - d)i$

    \item [Betrag Exponentialform:]
      $r = \sqrt{|a|^2 + |b|^2}$

    \item [Winkel Exponentialform:]
      $\phi = arctan(\frac{Im}{Re})$

    \item  [Multiplikation:]
      $(r_1 \cdot e^{i\phi_1}) \cdot (r_2 \cdot e^{i\phi_2}) = r_1 \cdot r_2 \cdot e^{i(\phi_1+\phi_2)}$
    
    \item [Disvision:]
      $\frac{r_1 \cdot e^{i\phi_1}}{r_2 \cdot e^{i\phi_2}} = \frac{r_1}{r_2} \cdot e^{i(\phi_1-\phi_2)}$
  \end{description}
\end{framed}