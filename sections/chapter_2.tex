\subsection*{Definitionen}

\begin{description}
  \setlength{\itemsep}{-.2cm}

  \item [Menge]
    Eine \textbf{Menge} ist eine Zusammenfassung 
    \textbf{wohlunterschiedenen Objekten} zu einem Ganzen.\\
    Es muss zudem einwandfrei entscheidbar sein, ob ein Objekt der Gesamtheit 
    angehört oder nicht.\\
    Die Objekte der Menge heißen \textbf{Elemente} der Menge.

  \item [gleichheit von Mengen:] 
    Zwei Mengen $A$ und $B$ heißen \textbf{gleich} ($A = B$) genau dann, wenn 
    jedes Element aus $A$ auch ein Element aus $B$ ist -- und umgekehrt.\\
    für alle $x \in A$ gilt auch $x \in B$ und für alle $y \in B$ gilt auch 
    $y \in A$ \\
    Sind die Mengen $A$ und $B$ nicht gleich, notiert man $A \neq B$.

  \item [Komplementärmenge:] 
    Sei $E(x)$ eine Aussageform über der Menge $U$.\\
    Dann heißen die Mengen 
    $M = \{x \in U | E(x)\}$ und $\bar{M} = \{x \in U | \neg E(x)\}$ in $U$ 
    komplemetär.\\
    $\bar{M}$ heißt \textbf{Komplementärmenge} oder \textbf{Komplement} von 
    $M$ in $U$.

  \item [Leere Menge:] 
    Die Menge, die kein Element enthält, heißt \textbf{leere Menge} und wird mit 
    $\emptyset$ bezeichnet.

  \item [Teilmenge:] 
    Eine Menge $B$ heißt \textbf{Teilmenge} einer Menge $A$ genau dann, wenn 
    jedes Element von $B$ auch ein Element von $A$ ist. ($B \subseteq A$ gilt 
    gdw. $\forall x : x \in B \rightarrow x \in A$) .\\
    $A$ heißt dann \textbf{Obermenge} von $B$. $B$ heißt echte Teilmenge von $A$.

  \item [Potenzmenge:] 
    Sei $M$ eine Menge. Die Menge aller Teilmengen von $M$ heißt 
    \textbf{Potenzmenge} von $M$ und wird $\mathcal{P}(M)$ notiert: \\
    $\mathcal{P}(M) := \{X | X \subseteq M\}$

  \item [Vereinigung:] 
    Seien $M$ und $N$ Mengen. Die Vereinigungsmenge ist Definiert durch: 
    $M \cup N := \{x | x \in M \text{ }oder\text{ } x \in N\}$

    %%% TODO: this should be fancier! UND/ODER should have space befire and after.
  \item [Schnitt:] 
    Seien $M$ und $N$ Mengen. Der Schnitt ist Definiert als 
    $M \cap N := \{x | x \in M \text{ }und\text{ } x \in N\}$

  \item [Differenz:] 
    Seien $M$ und $N$ Mengen. Die Differenz ist definiert durch 
    $M | N := \{x | x \in M \text{ }und\text{ } x \notin N\}$

  \item [Mengenfamilie:] 
    Die Mengenfamilie $\mathcal{F}$ sei definiert als 
    $\mathcal{F} \subseteq \mathcal{P}(M)$ \\
    Dann ist die Vereinigung aller Mengen aus $\mathcal{F}$:\\
    $\bigcup\mathcal{F} = \{x | \exists N \in \mathcal{P}(M) : (N \in \mathcal{F} \wedge x \in N)\}$ \\
    Der Durchschnitt aller Mengen aus $\mathcal{F}$: \\ 
    $\bigcap \mathcal{F} = \{x | \forall N \in \mathcal{P}(M) : (N \in \mathcal{F} \rightarrow x \in N)\} \cap \bigcup \mathcal{F}$ 

  \item [kartesisches Produkt:] 
    Seien $A$ und $B$ Mengen.\\
    Das \textbf{kartesische Produkt} (auch Kreuzprodukt) von $A$ und $B$ ist 
    definiert durch\\
    $A \times B := \{(a,b) | a \in A und b \in B\}$

  \item [Isomorphie:] 
    Zwei Mengen $X$ und $Y$ sind \textbf{isomorph}, wenn dich ihre Elemente eins
    zu eins zuordnen lassen. $(X \cong Y)$

  \item [Disjunkte Vereinigung:] 
    Die \textbf{disjunkte Vereinigung} von $A$ und $B$ ist definiert durch: \\
    $A \uplus B := (A \times {0}) \cup (B \times {1})$
\end{description}

  
\begin{framed} [Rechengesetze]
  \begin{tabular} {l | c}
    \thead[lc]{Assoziativgesetz:} & 
    \parbox{4.5cm}{\begin{equation*}
      \begin{aligned}
        (A \cup B) \cup C = A \cup (B \cup C) \\
        (A \cap B) \cap C = A \cap (B \cap C)
      \end{aligned}
    \end{equation*}}
    \\\hline
    \thead[lc]{Kommutativgesetz:} & 
    \parbox{3cm}{\begin{equation*}
      \begin{aligned}
        A \cup B = B \cup A \\
        A \cap B = B \cap A
      \end{aligned}
    \end{equation*}}
    \\\hline
    \thead[lc]{Distributivgesetz:} & 
    \parbox{5.3cm}{\begin{equation*}
      \begin{aligned}
        (A \cup B) \cap C = (A \cap C) \cup (B \cap C) \\
        (A \cap B) \cup C = (A \cup C) \cap (B \cup C)
      \end{aligned}
    \end{equation*}}
    \\\hline
    \thead[lc]{De-Morgan:} & 
    \parbox{4.5cm}{\begin{equation*}
      \begin{aligned}
        A | (B \cup C) = (A | B) \cap (A | C) \\
        A | (B \cap C) = (A | B) \cup (A | C)
      \end{aligned}
    \end{equation*}}
    \\\hline
    \thead[lc]{Absorption:} & 
    \parbox{3cm}{\begin{equation*}
      \begin{aligned}
        A \cap (A \cup B) = A \\
        A \cup (A \cap B) = A
      \end{aligned}
    \end{equation*}}
    \\\hline
    \thead[lc]{Idempotenz:} & 
    \parbox{2.2cm}{\begin{equation*}
      \begin{aligned}
        A \cap A = A \\
        A \cup A = A
      \end{aligned}
    \end{equation*}}
    \\\hline
    \thead[lc]{Komplement:} & 
    \parbox{2.2cm}{\begin{equation*}
      \begin{aligned}
        A \cap \bar{A} = \emptyset \\
        A \cup \bar{A} = G
      \end{aligned}
    \end{equation*}}
  \end{tabular}
\end{framed}

\vspace{-1cm}

\begin{framed} [Zahlenmengen]
  \begin{tabular} {l | c}
    \thead[lc]{natürliche Zahlen:} & 
    \parbox{.5\textwidth}{\begin{equation*}
      \begin{aligned}
        \mathbb{N} = {0,1,2,\dots}
      \end{aligned}
    \end{equation*}}
    \\\hline
    \thead[lc]{ganze Zahlen:} & 
    \parbox{.5\textwidth}{\begin{equation*}
      \begin{aligned}
        \mathbb{Z} = \{0,-1,1,-2,2,\dots\}
      \end{aligned}
    \end{equation*}}
    \\\hline
    \thead[lc]{rationale Zahlen:} & 
    \parbox{.5\textwidth}{\begin{equation*}
      \begin{aligned}
        \mathbb{Q} = \{\frac{m}{n} | m \in \mathbb{Z}, n \in \mathbb{N}, n \neq 0\}
      \end{aligned}
    \end{equation*}}
    \\\hline
    \thead[lc]{reelle Zahlen:} & 
    \parbox{.5\textwidth}{\begin{equation*}
      \begin{aligned}
        \mathbb{R}
      \end{aligned}
    \end{equation*}}
    \\\hline
    \thead[lc]{irrationale Zahlen:} & 
    \parbox{.5\textwidth}{\begin{equation*}
      \begin{aligned}
        \mathbb{R}\backslash \mathbb{Q}
      \end{aligned}
    \end{equation*}}
    \\\hline
    \thead[lc]{komplexe Zahlen:} & 
    \parbox{.5\textwidth}{\begin{equation*}
      \begin{aligned}
        \mathbb{C}
      \end{aligned}
    \end{equation*}}
    \\\hline
    \thead[lc]{Dabei gilt:} & 
    \parbox{.5\textwidth}{\begin{equation*}
      \begin{aligned}
        \mathbb{N} \subseteq \mathbb{Z} \subseteq \mathbb{Q} \subseteq \mathbb{R} \subseteq \mathbb{C}
      \end{aligned}
    \end{equation*}}
  \end{tabular}
\end{framed}

\vspace{-.8cm}

\begin{framed} [Schreibweisen Mengen]
  \begin{description}
    \item [explizit:] $M = \{1,2,3,4\}$
    \item [verbal:] \dq{}Die Menge aller nicht-negativen, geraden Ganzzahlen\dq{}
    \item [unendliche Mengen:] $M := \{0,2,4,6,8,\dots\}$
    \item [implizit:] $M := \{x \in \mathbb{N} | x ist gerade\}$
  \end{description}
\end{framed}

\vspace{-.8cm}

\begin{framed} [Potenzmenge:] 
  Jede mögliche Teilmenge einer Menge. Bsp: $M := \{1,2,3\}$ dann ist
  $\mathcal{P}(M) := \{\emptyset\},\{1\},\{2\},\{3\},\{1,2\},\{1,3\},\{2,3\},\{1,2,3\}$.
\end{framed}

\vspace{-.6cm}

\begin{framed} [disjunkten Vereinigung:] 
    Bei der disjunkten Vereinigung zweier Mengen kreuzt man die jeweiligen 
    Elemente mit einer Menge aus einem Index ($A \times {1}$, $B \times {2}$ usw.) 
    und vereinigt die daraus resultierenden Kreuzprodukte. 
    Dadurch ist jedem Element aus $A \uplus B$ anzusehen, ob es aus $A$ oder $B$ 
    stammt.
\end{framed}
