\section*{Definitionen}
\vspace{-0.3cm}

\textbf{1. Menge:} Eine \textbf{Menge} ist eine Zusammenfassung \textbf{wohlunterschiedenen Objekten} zu einem Ganzen.\\
Es muss zudem einwandfrei entscheidbar sein, ob ein Objekt der Gesamtheit angeh�rt oder nicht.\\
Die Objekte der Menge hei�en \textbf{Elemente} der Menge.

\textbf{2. gleichheit von Mengen:} Zwei Mengen \textit{A} und \textit{B} hei�en \textbf{gleich} (\textit{A = B}) genau dann, wenn jedes Element aus \textit{A} auch ein Element aus \textit{B} ist -- und umgekehrt.\\
f�r alle \textit{x $\in$ A} gilt auch \textit{x $\in$ B} und f�r alle \textit{y $\in$ B} gilt auch \textit{y $\in$ A}\\
Sind die Mengen \textit{A} und \textit{B} nicht gleich, notiert man \textit{A $\neq$ B}.

\textbf{3. Komplement�rmenge:} Sei \textit{E(x)} eine Aussageform �ber der Menge \textit{U}.\\
Dann hei�en die Mengen \textit{M = \{x $\in$ U | E(x)\}} und \textit{\begin{math}\bar{M}\end{math} = \{x $\in$ U | $\neg$E(x)\}} in \textit{U} komplemet�r.\\
\textit{\begin{math}\bar{M}\end{math}} hei�t \textbf{Komplement�renge} oder \textit{Komplement} vom \textit{M} in \textit{U}.

\textbf{4. Leere Menge:} Die Menge, die kein Element enth�lt, hei�t \textbf{leere Menge} und wird mit \textit{$\emptyset$} bezeichnet.

\textbf{5. Teilmenge:} Eine Menge \textit{B} hei�t \textbf{Teilmenge} einer Menge \textit{A} genau dann, wenn jedes Element von \textit{B} auch ein Element von \textit{A} ist. (\textit{B $\subseteq$ A gilt gdw. $\forall$x : x $\in$ B $\rightarrow$ x $\in$ A}) .\\
\textit{A} hei�t dann \textbf{Obermenge} von \textit{B}. \textit{B} hei�t echte Teilmenge von \textit{A}.

\textbf{6. Potenzmenge:} Sei \textit{M} eine Menge. Die Menge aller Teilmengen von \textit{M} hei�t \textbf{Potenzmenge} von \textit{M} und wird \textit{$\mathcal{P}$(M)} notiert: \\\textit{$\mathcal{P}$(M) := \{X | X $\subseteq$ M\}}

\textbf{7. Vereinigung:} Seien M und N Mengen. Die Vereinigungsmenge ist Definiert durch: \textit{M $\cup$ N := \{x | x $\in$ M oder x $\in$ N\}}

\textbf{8. Schnitt:} Seien M und N Mengen. Der Schnitt ist Definiert als \textit{M $\cap$ N := \{x | x $\in$ M und x $\in$ N\}}

\textbf{9. Differenz:} Seien M und N Mengen. Die Differenz ist definiert durch \textit{M$\textbackslash$ N := \{x | x $\in$ M und x $\notin$ N\}}

\textbf{10. Mengenfamilie:} Die Mengenfamilie $\mathcal{F}$ sei definiert als $\mathcal{F} \subseteq \mathcal{P}(M)$\\
Dann ist die Vereinigung aller Mengen aus $\mathcal{F}$:\\
$\bigcup\mathcal{F}$ = \{x | $\exists$ N $\in$ $\mathcal{P}$(M) : (N $\in$ $\mathcal{F} \wedge x \in N$)\}\\
Der Durchschnitt aller Mengen aus $\mathcal{F}:\\ \bigcap \mathcal{F} = \{x | \forall N \in \mathcal{P}(M) : (N \in \mathcal{F} \rightarrow x \in N)\} \cap \bigcup \mathcal{F}$ 

\textbf{11. kartesisches Produkt:} Seien \textit{A} und \textit{B} Mengen.\\
Das \textbf{kartesische Produkt} (auch Kreuzprodukt) von \textit{A} und \textit{B} ist definiert durch\\
\textit{A $\times$ B := \{(a,b) | a $\in$ A und b $\in$ B\}}

\textbf{12. Isomorphie:} Zwei Mengen \textit{X} und \textit{Y} sind \textbf{isomorph}, wenn dich ihre Elemente eins zu eins zuordnen lassen. \textit{(X $\cong$ Y)}

\textbf{13. Disjunkte Vereinigung:} Die \textbf{disjunkte Vereinigung} von \textit{A} und \textit{B} ist definiert durch:\\
\textit{A $\uplus$ B := (A $\times$ \{0\}) $\cup$ (B $\times$ \{1\})}



\vspace{3cm}
  
\section*{Rechengesetze}
\begin{equation}
\textbf{Assoziativgesetz:}
\begin{aligned}
(A \cup B) \cup C = A \cup (B \cup C) \\
(A \cap B) \cap C = A \cap (B \cap C) \notag
\end{aligned}
\end{equation}

\begin{equation}
\textbf{Kommutativgesetz:}
\begin{aligned}
A \cup B = B \cup A \\
A \cap B = B \cap A \notag
\end{aligned}
\end{equation}

\begin{equation}
\textbf{Distributivgesetz:}
\begin{aligned}
(A \cup B) \cap C = (A \cap C) \cup (B \cap C) \\
(A \cap B) \cup C = (A \cup C) \cap (B \cup C) \notag
\end{aligned}
\end{equation}

\begin{equation}
\textbf{De-Morgan:}
\begin{aligned}
A \textbackslash (B \cup C) = (A \textbackslash B) \cap (A \textbackslash C) \\
A \textbackslash (B \cap C) = (A \textbackslash B) \cup (A \textbackslash C) \notag
\end{aligned}
\end{equation}

\begin{equation}
\textbf{Absorption:}
\begin{aligned}
A \cap (A \cup B) = A \\
A \cup (A \cap B) = A \notag
\end{aligned}
\end{equation}

\begin{equation}
\textbf{Idempotenz:}
\begin{aligned}
A \cap A = A \\
A \cup A = A \notag
\end{aligned}
\end{equation}

\begin{equation}
\textbf{Komplementgesetze:}
\begin{aligned}
A \cap \bar{A} = \emptyset \\
A \cup \bar{A} = G \notag
\end{aligned}
\end{equation}

\vspace{-0.5cm}

\section*{Zahlenmengen}
\vspace{-0.3cm}
\textbf{natürliche Zahlen $\mathbb{N}$} = \{0,1,2,$\dots$\}\\
\textbf{ganze Zahlen $\mathbb{Z}$} = \{0,-1,1,-2,2,$\dots$\}\\
\textbf{rationale Zahlen $\mathbb{Q}$} = \{$\frac{m}{n}$ | m $\in \mathbb{Z}$ , n $\in \mathbb{N}$, n $\neq$ 0\}\\
\textbf{reelle Zahlen $\mathbb{R}$}\\
\textbf{irrationale Zahlen $\mathbb{R}\backslash\mathbb{Q}$ }\\
\textbf{komplexe Zahlen $\mathbb{C}$}\\

Dabei gilt: $\mathbb{N} \subseteq \mathbb{Z} \subseteq \mathbb{Q} \subseteq \mathbb{R} \subseteq \mathbb{C} \subseteq$

\section*{Schreibweisen Mengen:}
\vspace{-0.3cm}
Mengen können durch mehrere Schreibweisen beschrieben werden:\\
\textbf{explizit:} M = \{1,2,3,4\}\\
\textbf{verbal:} \glqq Die Menge aller nicht-negativen, geraden Ganzzahlen\grqq\\
\textbf{unendliche Mengen:} M := \{0,2,4,6,8,$\dots$\}\\
\textbf{implizit:} M := \{x $\in \mathbb{N}$ | x ist gerade\}\\

Eine \textbf{Potenzmenge} ist jede mögliche Teilmenge einer Menge. Bsp: M := \{1,2,3\} dann ist \\
$\mathcal{P}$(M) := \{\{$\emptyset$\},\{1\},\{2\},\{3\},\{1,2\},\{1,3\},\{2,3\},\{1,2,3\}\}.\\

Bei der \textbf{disjunkten Vereinigung} zweier Mengen kreuzt man die jeweiligen Elemente mit einer Menge aus einem Index (Menge A $\times$ \{1\}, Menge B $\times$ \{2\} usw.) und vereinigt die daraus resultierenden Kreuzprodukte. Dadurch ist jedem Element aus A $\uplus$ B anzusehen, ob es aus A oder B stammt.\\
