\subsection*{Definitionen}
\vspace{-0.3cm}

\begin{description}
  \item [Relation:] 
    Eine \textbf{Relation} zwischen $A$ und $B$ ist eine Teilmenge von 
    $A \times B$. Man schreibt $aRb$.\\
    $aRb \leftrightarrow (a, b) \in R$

  \item [Inverse Relation:] 
    Sei $R \subseteq A \times B$ eine Relation zwischen $A$ und $B$.
    Die \textbf{Inverse Relation} zu $R$ wird $R^{-1}$ notiert.\\
    $R^{-1} = \{(y,x) \in B \times A | (x,y) \in R\}$

  \item [Komposition:]
    Seien $R \subseteq M_1 \times M_2$ und $S \subseteq M_2 \times M_3$ 
    zweistellige Relationen.\\
    Die Verknüpfung $(R \circ S) \subseteq (M_1 \times M_3)$ heißt 
    \textbf{Komposition} der Relationen $R$ und $S$: \\
    $R \circ S := \{(x,z) | \exists y \in M_2 : (x,y) \in R \wedge (y,z) \in S\}$

  \item [Reflexivität:]
    Eine Relation $R \subseteq A^2$ über einer Menge $A$ heißt \textbf{reflexiv}, 
    wenn jedes Element in Relation zu sich selbst steht:\\
    $\forall a \in A : (a,a) \in R$

  \item [Symmetrie:]
    Eine Relation $R \subseteq A^2$ über einer Menge $A$ heißt 
    \textbf{symmetrisch}, wenn die Reihenfolge der Elemente keine Rolle spielt:\\
    $(a,b) \in R \rightarrow (b,a) \in R$
  
  \item [Antisymmetrie:]
    Das Gegenteil von Symmetrie:\\
    $(a,b) \in R \rightarrow \neg(b,a) \in R$

  \item [Antisymmetrie:] 
    Eine Relation R $\subseteq A^2$ über einer Menge $A$ heißt 
    \textbf{antisymmetrisch}, wenn aus der Symmetrie die Identität folgt: \\
    $((a,b) \in R \wedge (b,a) \in R) \rightarrow a = b$

  \item [Transitivität:] 
    Eine Relation $R \subseteq A^2$ über einer Menge $A$ heißt \textbf{transitiv}, 
    wenn aus einer Kette das mittlere Element entfernt werden kann:\\
    $((a,b) \in R \wedge (b,c) \in R) \rightarrow (a,c) \in R$

  \item [Totalität] 
    Eine Relation $R \subseteq A^2$ über einer Menge $A$ heißt \textbf{total} 
    (auch: linear), wenn je zwei Elemente in mindestens einer Richtung in 
    Relation stehen: \\
    $\forall (a,b) \in A : (a,b) \in R \vee (b,a) \in R$

  \item [Teilbarkeitsrelation:] 
    Sei $a \in \mathbb{N}^+, b \in \mathbb{Z}$.\\
    Wir schreiben $a|b$, wenn \dq{}a ein Teiler von b\dq{} ist, d.h. wenn gilt:\\
    $\exists k \in \mathbb{Z} : b = k \cdot a$

  \item [Rechtseindeutigkeit:]
    Eine Relation $R \subseteq A \times B$ heißt \textbf{rechtseindeutig} 
    (nacheindeutig) wenn Für alle $a \in A$ gilt:\\
    $((a,b) \in R \wedge (a,c) \in R) \rightarrow b = c$

  \item [Linkseindeutigkeit:]
    Eine Relation $R \subseteq A \times B$ heißt \textbf{linkseindeutig} wenn
    für alle $a \in B$ gilt: \\
    $((b,a) \in R \wedge (c,a) \in R) \rightarrow b = c$

  \item [Eindeutigkeit:]
    Eine Relation $R \subseteq A \times B$ heißt \textbf{eindeutig} wenn:\\
    R rechtseindeutig und R linkseindeutig.

  \item [Linkstotal:]
    Eine Relation $R \subseteq A \times B$ heißt \textbf{linkstotal} wenn: \\
    $\forall a \in A \exists b \in B : (a,b) \in R$

  \item [Rechtstotal:] 
    Eine Relation $R \subseteq A \times B$ heißt \textbf{rechtstotal}:\\
    $\forall b \in B \exists a \in A : (a,b) \in R$

  \item [Irreflexiv:]
    Eine Relation $R \subseteq A^2$ über einer Menge $A$ heißt 
    \textbf{irreflexiv} wenn:\\
    $\forall a \in A : (a,a) \notin R$

  \item [Alternativ:]
    Eine Relation $R \subseteq A^2$ über einer Menge $A$ heißt 
    \textbf{alternativ} wenn: \\
    $\forall a,b \in R : (a,b) \in R xor (b,a) \in R$

  \item [Äquivalenzrelation:]
    Ist eine Relation \textasciitilde{} reflexiv, symmetrisch und transitiv, 
    so wird sie \textbf{Äquivalenzrelation} genannt.

  \item [Äquivalenzklasse:]
    Gegeben sei eine Äquivalenzrelation $R$ über der Menge $A$. Dann ist Für 
    $a \in A: [a]_R = \{x | (a,x) \in R\}$ die \textbf{Äquivalenzklasse} von $a$.  

  \item [Zerlegung:]
    Sei $A$ eine nichtleere Menge. Eine \textbf{Zerlegung} (oder 
    \textbf{Partition}) von $A$ ist eine Mengenfamilie 
    $\mathcal{Z} \subseteq \mathcal{P}(A)$ mit:
    \begin{enumerate}
      \item Überdeckung: $A \subseteq \bigcup \mathcal{Z}$
      \item $\emptyset \notin \mathcal{Z}$
      \item Disjunktivität: $\forall M_1, M_2 \in \mathbb{Z}$ gilt entweder 
            $M_1 = M_2 \vee M_1 \cap M_2 = \emptyset$
    \end{enumerate}
    Eine Zerlegung ist also eine Einteilung von $A$ in nicht leere, paarweise 
    elementfremde Teilmengen, deren Vereinigung mit $A$ übereinstimmt.

  \item [Abschluss:]
    Sei $R$ eine Relation über $A$ und sei $\phi$ (reflexiv, usw.) eine 
    Eigenschaft von Relationen.\\
    Die Relation $R^*$ heißt \textbf{Abschluss} von $R$ bezüglich $\phi$, wenn 
    gilt:
    \begin{enumerate}
      \item $R*$ besitzt die Eigenschaft $\phi$
      \item $R \subseteq R*$
      \item Für alle Relationen $D$, die $R$ umfassen und ebenfalls die 
            Eigenschaft $\phi$ besitzen, gilt $R* \subseteq S$
    \end{enumerate}
    Mit anderen Worten: $R*$ ist die kleinste Relation, die $R$ umfasst und die 
    Eigenschaft $\phi$ besitzt.\\
    Besitzt $R$ bereits die Eigenschaft, so fügt der Abschluss nichts hinzu: 
    $R* = R$

  %		\textbf{22. Reflexiver Abschluss:} \\r(R) von R $\subseteq$ A $\times$ A:
  %		r(R) = R $\cup$ \{(a,a) | a $\in$ A\}
  %		
  %		\textbf{23. Symmetrischer Abschluss:} \\s(R) = R $\cup$ \{(b,a) | (a,b) $\in$ R\}
  %		
  %		\textbf{24. Transitiver Abschluss} \\t(R) = R$\cup$ \{(a,c) | es existiert $_{b_1,b_2,\dots,b_n}$ mit (a,$b_1$), ($b_i$,$b_{i+1}$), ($b_n$,$c_1$) $\in$ R\}

  \item [Ordnung:] 
    Ist eine Relation $\leq$ reflexiv, antisymmetrisch und 
    transitiv, so ist sie eine \textbf{Halbordnung} (partielle Ordnung)\\
    $a$ und $b$ heißen \textbf{vergleichbar} bzgl. $\leq$, falls 
    $a \leq b \vee b \leq a$ gilt (sonst unvergleichbar).\\
    Ist eine Halbordnung zusätzlich total, heißt die \textbf{(totale) Ordnung} 
    und $A$ heißt durch $\leq$ geordnet.

  \item [minimal/maximal:]
    Sei $\leq$ eine Halbordnungsrelation auf $A$. Sei $M$ eine nichtleere 
    Teilmenge von $A$.\\
    Ein Element $m \in M$ heißt \textbf{maximal} Element in $M$, wenn:\\
    $\forall m' \in M : m \leq m' \Rightarrow m = m'$\\
    $m$ heißt \textbf{minimales} Element, wenn:\\
    $\forall m' \in M : m' \leq m \Rightarrow m = m'$
\end{description}

\subsection*{Spezielle Relationen}
Es gibt spezielle Relationen, die im weiteren auch noch genutzt werden:\\
\begin{tabular} {l l}
  \textbf{Leere Relation:} & $R = \emptyset$\\
  \textbf{All-Relation:} &  $R = A \times B$\\
  \textbf{Identität (über M):} & $R = Id_M = \triangle_M := \{(x,x) | x \in M\}$
\end{tabular}


%		\textbf{Sätze}. Sei R eine Relation über A.
%		\vspace{-0.15cm}
%		\begin{itemize}
%			\setlength\itemsep{-0.05cm}
%			\item Dann ist R reflexiv genau dann, wenn $\triangle_A \subseteq$ R.
%			\item R ist Symmetrisch genau dann, wenn $R^{-1}$ dies ist. 
%			\item R ist transitiv genau dann, wenn R $\circ$ R $\subseteq$ R.
%			\item R ist antisymmetrisch genau dann, wenn R $\cap R^{-1} \subseteq \triangle_A$ 
%		\end{itemize}
%	
%		Der \textbf{Abschluss} einer Relation ist das Hinzufügen von einer Eigenschaft (reflexivität usw.), durch vereinigen mit den jeweils fehlenden Tupeln.
%		
%		Eine Äquivalenzrelation über der Menge A erzeugt immer eine Zerlegung von A. Gleichzeitig bestimmt jede Zerlegung von A immer eine Äquivalenzrelation. Dies wird am Beispiel der Restklasse Modulo (2) deutlich:\\
%		$[0]$ = \{0,2,4,6,$\dots$\}\\ 
%		$[1]$ = \{1,3,5,6,$\dots$\}\\
%		Dies sind Äquivalenzklassen -- Gleichzeitig also auch eine Zerlegung der Menge der natürliche Zahlen $\mathbb{N}$.
%		Diese Äquivalenzklassen können dann wieder vereinigt werden, und man erhält die ursprüngliche Menge--hier also die Menge der natürliche Zahlen $\mathbb{N}$
%		
%		


%		\section*{Zusammenfassung}
%		\vspace{-0.2cm}
%		Eine Relation ist eine Menge aus Paaren in der Form (a,b) aus einem Kreuzprodukt A $\times$ B. Die Menge R enthält nur die Paare, die durch diese Beziehung verknüpft sind $\rightarrow$ Teilmenge.\\
%		Schreibweisen Für Relationen sind:\\
%		\textbf{Infix-Notation:} x$\mathcal{R}$y\\
%		\textbf{Präfix Notation:} $\mathcal{R}$xy\\
%		\textbf{Postfix-Notation:} xy$\mathcal{R}$\\
%		
%		\vspace{-0.25cm}
%		
%		Beispiele Für Relationen sind z.B:\\
%		\vspace{-0.4cm}
%		\begin{itemize}
%			\setlength\itemsep{-0.05cm}
%			\item $< und \leq$ Relationen auf $\mathbb{N} \times \mathbb{N}$
%			\item $P_3$ := \{(x,x+3) | x $\in \mathbb{N}^+$\}
%		\end{itemize}
%		
%		Bei der \textbf{Komposition von Relationen} werden zwei Relationen zusammengefasst. geschrieben: R $\circ$ S. Dies ist dann die Teilmenge aus $M_1 \times M_3$\\
%		Für komponierte Relationen gelten die Rechengesetze genauso, wie Für Mengen. (siehe Rechengesetze VL 2)\\
%		
%		\vspace{-0.25cm}
