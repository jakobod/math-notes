\begin{framed} [Direkter Beweis]
  Wird bei wenn-dann-Aussagen genutzt. Modus ponens: Aus $p$ und 
  $(p \Rightarrow q)$ ergibt sich $s$. Vorgehen dabei ist:
  \begin{enumerate}
    \item Satz genau studieren--Welche Parameter werden gestellt?
    \item Bei der Hypothese beginnen. Diese muss wahr sein, denn 
          $(p \Rightarrow q)$.
    \item ggf. die Hypothese mathematisch darstellen Bsp:\\
          gerade Zahl $n = 2k$\\
          ungerade Zahl $n = 2k + 1$
    \item Dann durch (beliebig viele) Folgeaussagen von der Hypothese zur 
          Schlussfolgerung kommen.\\
          $p \Rightarrow s_1, s_1 \Rightarrow s_2, s_2 \Rightarrow q$ wobei 
          $s_1 - s_n$ wieder wahre Aussagen sind.
    \item Praktisch dabei: Es muss nicht jeder Schritt aufgeschrieben werden--nur
          solche, die wichtig für die Beweisführung des Lemma sind.
  \end{enumerate}

  \subsubsection*{Beispiel:}
  \textbf{Satz:} \dq{}Die Summe von drei aufeinander folgenden natürlichen Zahlen 
                ist durch drei teilbar.\dq{}\\
  \textbf{Gegebene Informationen: $n \in \mathbb{N}$ und\\
  $p = n + (n+1) + (n+2)$ ist durch 3 teilbar}
  \begin{enumerate}
    \item $n \in \mathbb{N} \Rightarrow (n+1) \in \mathbb{N}$
    \item $\Rightarrow n + (n+1) + (n+2) = (3n + 3)$
    \item $\Rightarrow (3n + 3) = 3(n + 1)$
  \end{enumerate}
  Damit ist der Satz bewiesen. $\square$
\end{framed}


\begin{framed} [Kontraposition]
  Ist dem Direkten Beweis sehr ähnlich. Nur dass hier die Behauptung negiert und
  umgekehrt wird um zu der äquivalenten Kontraposition zu gelangen. Aus 
  $(p \Rightarrow q)$ wird also $(\neg q \Rightarrow \neg p)$. Man beweist also 
  quasi rückwärts.

  \subsubsection*{Beispiel:}
  \textbf{Satz:} \dq{}Wenn $a^2$ eine ungerade Zahl ist, dann ist $a$ ungerade\dq{}.\\
  Die äquivalente Kontraposition dazu ist:
  \dq{}Wenn $a$ gerade, dann ist $a^2$ gerade\dq{}.
  \begin{enumerate}
    \item $\neg q =$ \glqq $a$ ist gerade\grqq
    \item $a = 2 \cdot k$ (Def. gerade Zahl)
    \item $a \cdot a = (2 \cdot k) \cdot a$ (mul. mit a)
    \item $a^2 = 2 \cdot (k \cdot a)$ (Assoziativgesetz)
    \item $a^2 = 2 \cdot k'$ (wobei $k' = a \cdot k$)
    \item $\neg p = a^2$ ist gerade (durch $2 \cdot k'$)
  \end{enumerate}
  Damit ist der Satz bewiesen. $\square$
\end{framed}


\begin{framed} [Wiederspruch]
  Basiert wieder auf der Implikation ($p \Rightarrow q$). Hier wird aber ein 
  Wiederspruch erzeugt sodass $(p \wedge \neg q)$

  \subsubsection*{Beispiel:}
  \textbf{Satz:} \dq{}Wenn $a$ und $b$ gerade natürliche Zahlen sind, dann ist
  auch $a \cdot b$ gerade\dq{}.
  \begin{enumerate}
    \item Annahme: $a \cdot b$ ist ungerade.
    \item $a \cdot b = 2 \cdot (a \cdot k)$ (denn: $b = 2 \cdot k$)
    \item $a \cdot k$ ist gerade. Also muss $a \cdot b$ gerade sein.
  \end{enumerate}
  Damit ist der Satz bewiesen. $\square$
\end{framed}


\begin{framed} [Äquivalenzbeweis]
  Bei dieser Beweistechnik unterteilt man die Aussage in zwei direkte Beweise. Aus
  $(p \Leftrightarrow q)$ wir dann $(p \Rightarrow q)$ und $(q \Rightarrow p)$.

  \subsubsection*{Beispiel:}
  \textbf{Satz:} \dq{}$a$ ist gerade genau dann, wenn $a^2$ gerade ist\dq{}.\\
  Dabei ist $p = $\dq{}$a$ ist gerade\dq{} und $q = \dq{}a^2$ ist gerade\dq{}\\
  In diesem Fall ist $(p \Rightarrow q)$ schon bewiesen. (siehe Bsp. Kontraposition)\\
  $(q \Rightarrow p)$ wird durch Kontraposition bewiesen:
  \begin{enumerate}
    \item $\neg p$: \glqq $a$ ist ungerade\grqq
    \item $a - 1 = 2 \cdot k$ (def. ungerade Zahl umgestellt)
    \item $a = 2 \cdot k + 1$
    \item $a^2 = (2 \cdot k)^2 + 2 \cdot (2 \cdot k) + 1$ (quadrat schon ausmul.)
    \item $a^2 = 2 \cdot (2 \cdot k \cdot k + 2 \cdot k) + 1$
    \item $a^2$ ist ungerade
  \end{enumerate}
  Da nun sowohl ($p \Rightarrow q$) als auch ($q \Rightarrow p$) bewiesen ist, ist
  der Äquivalenzbeweis erbracht. $\square$
\end{framed}


\begin{framed} [Fallunterscheidung]
  Jede Aussage $p$ ist logisch äquivalent zu 
  $(q \Rightarrow p) \wedge (\neg q \Rightarrow p)$. Dann Beweist man einfach
  beide Fälle.

  \subsubsection*{Beispiel:}
  \textbf{Satz:} \dq{}Jede natürliche Zahl $n^2$ geteilt durch 4 lässt entweder
  den Rest $1$ oder $0$\dq{}.
  \begin{description}
    \item [$n$ ist gerade:] \hfill
      \begin{itemize}
        \item $n = 2m$ für $m \in \mathbb{M}$
        \item $n^2 = 4m^2$
        \item $n^2$ ist durch $4$ teilbar
        \item Rest ist $0$
        \item Rest ist $1$ oder $0$
      \end{itemize}
    \item [$n$ ist ungerade:] \hfill
      \begin{itemize}
        \item $n = 2m + 1$ für $m \in \mathbb{M}$
        \item $n^2 = 4m^2 + 4m + 1 = 4(m^2 + m) + 1$
        \item $n^2$ ist durch $4$ teilbar mit Rest $1$
        \item Rest ist $1$
        \item Rest ist $1$ oder $0$
      \end{itemize}
  \end{description}
  Damit sind alle Fälle betrachtet und die Aussage bewiesen. $\square$
\end{framed}


\begin{framed} [Beweis mit Quantoren]
  Bei universellen Aussagen ($\forall x$) muss man unabhängig von konkreten 
  Werten für die Quantifizierten Variablen Beweisen. Deswegen beginnt und 
  beendet man den Beweis etwas anders:\\
  Bei der Aussage $\forall x : (p(x) \Rightarrow q(x))$ würde man so vorgehen:
  \begin{enumerate}
    \item Sei $a$ ein beliebiger, aber fester Wert aus dem Universum 
          (also der Menge).
    \item <Beweis>
    \item Da $a$ beliebig gewählt werden kann, folgt:\\
          $\forall x : (p(x) \Rightarrow q(x))$.
  \end{enumerate}
  Damit ist die Aussage Bewiesen $\square$
\end{framed}


Bei existenziellen Aussagen $\exists x : (p(x) \Rightarrow q(x))$
geht man so vor:
\begin{enumerate}
  \item Sei $a$ = <Ein geeignetes Element aus dem Universum>.
  \item <Beweis>
  \item Damit ist die Existenz eines $a$ mit der Eigenschaft ($p(x) \Rightarrow q(x)$) bewiesen. 
  \item Damit ist die Gültigkeit der Aussage $\exists x : (p(x) \Rightarrow q(x))$ bewiesen.
\end{enumerate}
