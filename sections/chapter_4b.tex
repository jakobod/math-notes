%\small %oder \tiny oder \scriptsize oder \footnotesize oder \small ...
\subsection*{Definitionen}
\begin{description}
  \item [Abbildung:] 
    Unter einer Abbildung $f$ von einer Menge $A$ in einer Menge $B$ versteht 
    man eine Vorschrift, die jedem $a \in A$ eindeutig ein bestimmtes 
    $b = f(a) \in B$ zuordnet.
    \begin{itemize}
      \setlength\itemsep{-0.05cm}
      \item Schreibweise: $f : A \rightarrow B$.
      \item für die Elementzuordnung verwendet man die Schreibweise 
            $a \rightarrowtail b = f(a)$
      \item Man bezeichnet $b$ als das Bild von $a$.
      \item $a$ ist ein Urbild von $b$
    \end{itemize}

  \item [Abbildung:] 
    Sei $F \subseteq A \times B$ eine linksvollständige und rechtseindeutige 
    Relation.
    \begin{enumerate}
      \item $F$ ist linksvollständig: für alle $a \in A$ gilt: Es existiert ein 
            $b \in B$, so dass $(a,b) \in R$
      \item $F$ ist rechtseindeutig: für alle $a \in A$ und alle 
            $b_1, b_2 \in B$ gilt: $(a,b_1) \in R$ und $(a,b_2) \in R$, dann 
            $b_1 = b_2$
    \end{enumerate}
    Das Tripel $f = (A,B,F)$ heißt Abbildung von A nach B.
    \begin{itemize}
      \item $F$ heißt Graph der Abbildung
      \item $A$ ist der Definitionsbereich
      \item $B$ ist der Bildbereich
    \end{itemize}
    Zu jedem $a \in A$ wird das eindeutig bestimmte $b \in B$ mit $aFb$ als Bild
    von $f$ bei $a$ bezeichnet. Notation: $f(a)$

  \item [Bild:]
    Sei $f : A \rightarrow B$ und $M \subseteq A$
    \begin{itemize}
      \item Das Bild von $M$ unter $f$ ist die Menge:
            $f(M) := \{f(x) | x \in M\}$
      \item Insbesondere heißt $Bild(f) := f(A)$ das (volle) Bild von $f$
            (auch Wertebereich).
      \item Das Urbild einer Teilmenge $N \subseteq B$ ist definiert durch: 
            $f^{-1}(N) := \{a\in A | f(a) \in N\}$
    \end{itemize}

  \item [Einschränkung:]
    Sei $f = (A,B,F)$ eine Abbildung und $M \subseteq A$.\\
    Die Abbildung $f|_m = (M,B,F \cap (M \times B))$ heißt Einschränkung von
    $f$ auf $M$.

  \item [Komposition:]
    Kompositionen von Funktionen ist hier definiert als: 
    $a \rightarrowtail (g \circ f)(a) = g(f(a)), a \in A$

  \item [Injektiv:]
    Wenn für alle $a,a' \in A$ mit $a \neq a'$ gilt $f(a) \neq f(a')$.

  \item [surjektiv:]
    Falls es für jedes $b \in B$ ein $a \in A$ gibt mit $f(a) = b$.

  \item [bijektiv:]
    Falls $f$ sowohl injektiv als auch surjektiv ist.

  \item [Inverse Abbildung:]
    Sei $f : A \rightarrow B$ eine bijektive Abbildung. Da existiert zu $f$ 
    stets eine Abbildung $g$ mit $g \circ f = id_A$ und $f \circ g = id_B$.
    $g$ heißt die zu $f$ inverse Abbildung oder Umkehrabbildung. Notation: 
    $f^{-1}$.

  \item [Gleichmächtig:]
    Seien $M$ und $N$ zwei Mengen. $M$ und $N$ heißen gleichmächtig 
    (oder umfangsgleich) genau dann, wenn es eine bijektive Abbildung 
    $f : M \rightarrow N$ gibt. Notation $M \tilde{=} N$. ($|M| = |N|$)

  \item [endlich:]
    Eine Menge $M$ heißt endlich genau dann, wenn $M = \emptyset$ oder es für
    ein $n \in \mathbb{N}$ eine bijektive Abbildung
    $b : M \rightarrow \mathbb{N}_n$ gibt.

  \item [unendlich:]
    Eine Menge $M$ heißt unendlich genau dann, wenn $M$ nicht endlich ist.

  \item [abzählbar:]
    Eine Menge $M$ heißt abzählbar genau dann, wenn $M$ endlich ist oder es eine 
    bijektive Abbildung $b : M \rightarrow \mathbb{N}$ gibt.

  \item [abzählbar unendlich:]
    Eine Menge $M$ heißt abzählbar unendlich genau dann, wenn $M$ abzählbar und
    unendlich ist.

  \item [überabzählbar:]
    Eine Menge heißt überabzählbar genau dann, wenn $M$ nicht abzählbar ist.

  \item [Folge:]
    Eine Folge Reeller Zahlen ist eine Funktion 
    $f : \mathbb{N} \rightarrow \mathbb{R}$.

  \item [konvergenz:]
    Eine Folge ($a_n$) konvergiert gegen $a \in \mathbb{R}$, wenn gilt: 
    Zu jedem $\epsilon > 0$ existiert ein $N \in \mathbb{R}$, so dass gilt:\\
    $|a_n - a| < \epsilon$ für alle $n > N$\\
    Die Zahl $a$ heißt Grenzwert (Limes) der Folge ($a_n$). Eine Folge ($a_n$) 
    mit Grenzwert heißt konvergent. Man schreibt: 
    $lim_{n\rightarrow \infty} a_n = a$ oder auch $a_n \rightarrow a$ für 
    $n \rightarrow \infty$.\\
    Eine Folge, die gegen $a = 0$ konvergiert, heißt Nullfolge.

  \item [Reihe:]
    Sei ($a_n$) eine Folge. Die Reihe ($s_n$) ergibt sich aus ($a_n$) durch 
    Summation: $s_n := \sum_{k=0}^{n} a_n$

  \item [beschränkt(e Folge):]
    Eine Folge ($a_n$)heißt beschränkt, wenn es eine Zahl $s$ gibt, so dass 
    $|a_n| \leq s$ für alle $n$ gilt.

  \item [monoton wachsend:]
    Eine Folge ($a_n$) heißt monoton wachsend, wenn $a_n \leq a_{n+1}$ für alle 
    $n$ gilt.

  \item [monoton fallend:]
    Eine Folge ($a_n$) heißt monoton fallend, wenn $a_n \geq a_{n+1}$ für alle 
    $n$ gilt.

  \item [supremum:]
    Eine Zahl $s$ heißt Suprmum einer Menge $M \subseteq \mathbb{R}$, wenn $s$ 
    die kleinste obere Schranke von $M$ ist, d.h:
    \begin{itemize}
      \item $s$ ist obere Schranke von $M$ ($\forall m \in M : m \leq s$)
      \item jede Zahl $x < s$ ist keine obere Schranke von $M$
    \end{itemize}

  \item [infimum:]
    Eine Zahl $i$ heißt Infimum einer Menge $M \subseteq \mathbb{R}$, wenn $i$
    die größte untere Schranke von $M$ ist.

  \item [Häufungspunkt:]
    $h$ heißt Häufungspunkt einer Folge ($a_n$), wenn jede Umgebung 
    $K_\epsilon (h)$ von $h$ undendlich viele Folgeglieder enthält. Also:\\
    $|h - a_n| < \epsilon$ für unendlich viele $n$

  \item [Cauchy-Folge:]
    Eine Folge ($a_n$) heißt Cauchy-Folge, wenn es zu jedem $\epsilon > 0$ ein
    $n$ gibt so dass gilt: $|a_n -a_m| < \epsilon$, falls $n$ und $m$ > $N$ sind.

  \item [Asymptotisch:]
    Zwei Folgen ($a_n$) und ($b_n$) mit $b_n \neq 0$ heißen asymptotisch gleich,
    falls die Folge ($\frac{a_n}{b_n}$) gegen 1 konvergiert. Notation: 
    $a_n \tilde{-} b_n$.

  \item [O-Notation:]
    Sei $g : \mathbb{N} \rightarrow \mathbb{R}$. Eine Funktion 
    $f : \mathbb{N} \rightarrow \mathbb{R}$ gehört zu der Menge $O(g)$, wenn es
    eine Konstante $c \in \mathbb{R}^+ gibt, so dass |f(n)| \leq c \circ |g(n)|$
    für fast alle $n$ gilt.
\end{description}


\begin{framed} [Rechenregeln Folgen]
  \begin{itemize}
    \item $lim_{n\rightarrow \infty} \frac{1}{n^s} = 0$ für jedes positive $s \in \mathbb{Q}$.
    \item $lim_{n\rightarrow \infty} \sqrt[n]{a} = 1$ für jedes reelle $a > 0$.
    \item $lim_{n\rightarrow \infty} \sqrt[n]{n} = 1$.
    \item $lim_{n\rightarrow \infty} q^n = 0$ für jedes reelle $q$ mit $|q| < 1$.
  \end{itemize}
\end{framed}


%	\section*{Zusammenfassung}
%	Funktionen sind Relationen, die ein Urbild ($A$) mit einem Bild ($B$) verknüpfen. Funktionen sind dabei linksvollständig und rechtseindeutig. 
%	
%	Die Inverse Abbildung ist der Schritt vom Bild zurück zum Urbild. Dies lässt sich (wie bei der negation $\neg$) beliebig oft durchführen--Ergebnis ist immer Bild oder Urbild. 
%	
%	Eine Folge definiert sich durch die Folgenvorschrift. mana zählt ganzzahlige Schritte hoch und errechnet dann anhand der Vorschrift die jeweiligen Folgenglieder. 
%	Die Reihe ist eine weiterführung der Folge. Diese Summiert die jeweiligen Folgenglieder bis stelle $n$ auf um die Reihenglieder zu bestimmen. 
