\documentclass[ngerman, 9pt, twocolumn, DIV=20, headings=small]{scrartcl}


\usepackage{babel, lmodern}
\usepackage[T1]{fontenc}
\usepackage[latin1]{inputenc}%oder was immer 
\usepackage{amstext}
\usepackage{amsmath}
\usepackage{amssymb}
\usepackage{multicol}
\usepackage{caption}
\usepackage{scrpage2}
\pagestyle{scrheadings}

%\renewcommand{\labelitemi}{$\Rightarrow$}
%\renewcommand{\labelitemii}{$\Rightarrow$}

\pagenumbering{arabic}

\begin{document}

\chead{Jakob Otto --- Zusammenfassung Kap. 6: Induktion --- Stand: \today}


%\small %oder \tiny oder \scriptsize oder \footnotesize oder \small ...
	\section*{Definitionen}
	\textbf{1. Induktion:} Ist $A(n)$ eine von $n \in \mathbb{N}$ abh�ngige Aussage, so sind dazu die folgenden beiden Beweisschritte 1 und 3 durchzuf�hren:
	\begin{itemize}
		\item Induktionsanfang (IA): Man zeigt, dass $A(0)$ richtig ist.
		\item Induktionsbehauptung (IB): Annahme, dass $A(k)$ f�r ein festes $k$ richtig ist.
		\item Induktionsschluss (IS): Man zeigt: Aus der Annahme, dass $A(k)$ richtig ist (Induktionsanker), folgt, dass auch $A(k + 1)$ richtig ist:\\
		$A(k) \Rightarrow A(k + 1)$
	\end{itemize}
	Dann ist gew�hrleistet, dass $A(n)$ f�r alle $n \in \mathbb{N}$ gilt.\\
	\textbf{Wichtig:} Wird der Induktionsanfang nicht f�r $n_0 = 0$, sondern f�r ein $n_0 > 0$ durchgef�hrt, so gilt die Aussage nur f�r alle $n \geq n_0$.
	
	\textbf{2. Induktiv erzeugte Menge:} Die Menge $M$ wird wie folgt induktiv definiert:
	\begin{itemize}
		\item Basismenge: jedes $x$ mit $\tilde{}$ geh�rt zu $M$.
		\item Erzeugungsregel: Sind $m_1$ und $m_2$ Elemente aus $M$, dann auch das Element $m = randoomFunction(m_1,m_2)$.
		\item Nur Elemente, die so gebildet werden k�nnen, geh�ren zu $M$.
	\end{itemize}

	\textbf{3. W�rter/Zeichenketten:} Sei $\sum$ ein Alphabet. Die Menge $\sum^*$ aller W�rter �ber $\sum$ ist induktiv definiert:
	\begin{itemize}
		\item Basismenge: Das leere Wort $\epsilon$ geh�rt zu $\sum^*$; das hei�t: $\epsilon \in \sum^*$.
		\item Erzeugungsregel: Ist $w$ ein Wort in $\sum^*$ und $a$ ein Element von $\sum$, dann geh�rt die Konkatenation $wa$ zu $\sum^*$
	\end{itemize}

	\textbf{4. L�nge eines Wortes:} Sei $\sum$ ein Alphabet. Die L�nge eines Wortes $w \in \sum^*$ ist induktiv definiert durch:
	\begin{itemize}
		\item Die L�nge des leeren Wortes $\epsilon$ ist $0$, |$\epsilon| = 0$
		\item Sei $w \in \sum^*$ und $a \in \sum$. Dann ist $|wa| = |w| + 1$. 
	\end{itemize}

	\textbf{5. aussagenlogische Formeln:} Sei $X$ die Menge der aussagenlogischen Variablen.\\
	Die Menge der aussagenlogischen Formeln wird wie folgt definiert:
	\begin{itemize}
		\item Basismenge: Die Konstanten $w$ und $f$ sind aussagenlogische Formeln.
		\item Basismenge: Jede Aussagenlogische Variable $x \in X$ ist eine aussagenlogische Formel.
		\item Erzeugungsregel: Sind $\alpha$ und $\beta$ aussagenlogische Formeln, so sind auch $(\neg \alpha), (\alpha \wedge \beta), (\alpha \vee \beta), (\alpha \Rightarrow \beta) und (\alpha \Leftrightarrow \beta)$ aussagenlogische Formeln.
	\end{itemize}
	
	\textbf{Beispiel Induktion:}
	Satz: F�r alle nat�rlichen Zahlen $n$ gilt $\sum_{i=0}^{n} i = \frac{1}{2}n(n + 1)$.
	\begin{itemize}
		\item Induktionsanfang (IA): Die Eigenschaft gilt f�r $n = 0$, denn $\sum_{i=1}^{n} i = 0$ und $\frac{1}{2}n(n + 1) = 0$.
		\item Induktionsbehauptung (IB): Wir nehmen an, dass die Summenformel f�r ein beliebiges, aber festes $k$ gilt: $\sum_{i=0}^{k} i = \frac{1}{2}k(k + 1)$.
		\item Induktionsschluss (IS): Unter der Vorraussetzung, dass die IB gilt, wollen wir die Summenformel f�r $k + 1$ zeigen: $\sum_{i=1}^{k+1}i = \frac{1}{2}k(k + 1)(+2)$ Dies k�nnen wir durch folgende Umformung zeigen:\\
		$\sum_{i=1}^{k+1}i = (\sum_{i=0}^{k})+(k+1) = \frac{k(k+1)}{2}+(k+1) = \frac{k(k+1)+(2k+2)}{2} = \frac{k^2+3k+2}{2} = \frac{(k+1)(k+2)}{2}$
	\end{itemize}
	Nach dem Induktionsprinzip gilt die Aussage also f�r alle nat�rlichen Zahlen.
	
	\textbf{Verallgemeinerte vollst�ndige Induktion:}
	Diese wird genutzt, wenn die Induktionsannahme $A(n)$ nicht genug ist, um den Induktionsschluss $A(n+1)$ beweisen zu k�nnen.\\
	Die Aussage $A(n)$ gilt f�r alle nat�rlichen Zahlen, wenn sowohl der Induktionsanfang $A(0)$ als auch der Induktionsschritt gilt:
	\begin{center}
		$\forall n \in \mathbb{N} : (A(0) \wedge \dots \wedge A(n)) \Rightarrow A(n+1)$
	\end{center}

	\textbf{Beispiel:} Sei $n$ eine nat�rliche Zahl und $n \geq 2$. Dann ist $n$ das Produkt von Primzahlen. \\
	\begin{itemize}
		\item \textbf{IA:} Die Eigenschaft gilt f�r $n = 2$, denn $2$ ist das Produkt von sich selbst, also einer Primzahl.
		\item \textbf{Starke Induktionsbehauptung (IB):} F�r festes $n$ nehmen wir an, dass sich alle Zahlen $2,3,\dots, n$ als Produkt von Primzahlen schreiben lassen.
		\item \textbf{Induktionsschluss (IS):} Z.z.: $n + 1$ ist ein Produkt von Primzahlen.\\
		Fall 1: $n + 1$ ist eine Primzahl, dann auch ein Produkt von Primzahlen (sich selbst).\\
		Fall 2: $n + 1$ ist keine Primzahl, dann gibt es mindestens zwei echte Teiler $b$ und $c$:\\
		Da $b$ und $c$ beide echt kleiner als $n + 1$ sind, gilt die IB f�r sie:
		\begin{center}
			$b = p_1\dots p_k \space\space c = q_1\dots q_l$
			
		\end{center}
		Damit gilt: $n + 1 = b \cdot c = (p_1\dots p_k)\cdot(q_1\dots q_l)$\\
		$n + 1$ ist also ein Produkt von Primzahlen.
	\end{itemize}
	Nach dem verallgemeinerten Induktionsprinzip gilt die Aussage also f�r alle nat�rlichen Zahlen $n\geq 2$.
	
	
	\section*{\glqq Induktionsschablone\grqq}
	\begin{itemize}
		\item \textbf{Induktionsanfang (IA):} Die Eigenschaft gilt f�r $n = 0$, denn $\dots$
		\item \textbf{Induktionsbehauptung (IB):} Wir nehmen an, dass die Summenformel f�r ein beliebiges, aber festes $k$ gilt: $\sum_{i=0}^{k} \dots$
		\item \textbf{Induktionsschluss (IS):} Unter der Voraussetzung, dass die IB gilt, wollen wir die Summenformel f�r $k + 1$ zeigen: $\sum_{i=0}^{k+1} \dots$
	\end{itemize}
	Nach dem Induktionsprinzip gilt die Aussage also f�r alle nat�rlichen Zahlen.
	
	\textbf{Peano-Axiome:}
	\vspace{-0.3cm}
	\begin{center}
		$0 \cdot m = 0$\\
		$(n + 1) \cdot m = (n \cdot m) + m$\\
		$n \cdot (m_1 + m_2) = n \cdot m_1 + n \cdot m_2$
	\end{center}
	
	
	
	
\end{document}